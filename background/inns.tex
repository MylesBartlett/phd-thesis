% % ********************************************************************************
\section{Invertible neural networks}\label{sec:inns}
% % ********************************************************************************
Chapter 3 of this thesis explores the application of invertible neural networks (INNs) to
fair-representation learning and so will afford some brief discussion to their basic workings here.
%
\marginpar{\textbf{Bijectivity}}
%
An INN \citep{kobyzev2020normalizing}, as the name suggests, refers to any neural network for which
both the usual forward mapping, \(f(\cdot)\), and its inverse \(f(\cdot)\) are defined, with the
assumed property that both are differentiable and as such that the function belongs to the class of
\emph{diffeomorphisms}, \(f \in \text{Diff}(\gX)\). 
%
Thus, we have a function that is invertible bicontinuous map from input space, \(\gX \subset \R^d
\) to latent space \(\gZ \subset \R^d \), noting that the domain and codomain are equidimensional,
as presupposed by the function's bijectivity.
%
It is obvious, but nonetheless worth stating, that for \(f\) is composed of subfunctions \(f
\triangleq f_L \circ \dots \circ f_2 \circ f_1\) and each individual subfunction is diffeomorphic,
then \(f\) in its totality, also satisfies this property, allowing us to build arbitrarily complex
INNs by chaining together layers defining these subfunctions.
%

The usual bailiwick of INNs is density estimation -- and by complement, generative modelling --
due to their hallmark diffeomorphic property that make it possible for densities under the models
to be calculated \emph{exactly}, in contrast to variational methods that only do so up to a lower
bound (the so-called ELBO). 
%
\marginpar{\textbf{Normalising flows}}
%
This calculation is enabled by the change-of-variables theorem, allowing one to track how the
density of the distribution changes as the INN warps a known (and tractable) base distribution into
a complex, highly-multimodal, one.
%
Like with variational auto-encoders, the base density, \(P(Z)\), is generally taken to be an
Isotropic Gaussian distribution; the posterior density, \(P(X)\), `flows' through the network -- in
a manner reminiscent of a Galton Board -- into this normalised base density, earning this class of
methods the name \emph{normalising flows} (NFs; \cite{rezende2015variational,
kobyzev2020normalizing}).
%
Practically, for a given sample \(x\), its log-likelihood under the INN \(f\), with base density
\(\gN(\cdot; 0, \sI)\) the aforementioned Gaussian distribution, can be computed as
%
\begin{equation*}
%
    \begin{aligned}
        %
        &\log P(X=x) = 
        %
        \log P(Z=z) + \sum_{l=1}^L \log \left| \det\left( \frac{\diff f_l}{ f_{l-1}}\right)
        \right|, \\
        %
        &P(Z=z) = \gN(z; 0, \sI),
        %
    \end{aligned}
\label{eq:cov}
%
\end{equation*}
%
and training the model simply amounts to maximising this quantity over the empirical training
distribution in the usual fashion.
%
As with GANs and VAEs, to sample from \(P(X)\), one needs only draw a random sample from the
corresponding base density, \(z \sim P(Z) \), and push that sample through \(f\).


\marginpar{\textbf{Losslessness}}
%
In Chapter 3, however, it is not the foregoing density-estimation capabilities of INNs that we are
interested in, rather the diffeomorphic property itself, insofar as it guarantees the learned
representations are \emph{lossless} \wrt{} the inputs as well as a means to visualise the factors
of said representations due to its having an exact inverse (whereas auto-encoders have only an
approximate inverse (the decoder) that must be trained separately from the encoder).
%
That is to say, while \(f\) may deform manifold \(\gX\) in arbitrarily non-linear ways, since each
point is mapped uniquely from the domain to codomain then only the form of the information
contained in the input can change, not its extent. 
%
This is in contrast to conventional architectures that define \emph{surjective} mappings that embed
inputs into spaces much smaller than \(\R^d\) (in line with the \emph{Manifold Hypothesis}
\citep{fefferman2016testing}).
%
Other works have also capitalised on this information-preservation explicitly, e.g. both
\cite{hoogeboom2019integer} and \cite{xie2021enhanced} explore the natural suitability of INNs for
lossless image compression.
%
Contrastingly, in work postdating that done in Chapter 3, normalising flows have been applied
applied to FairML problems with the insight that one can leverage the exact-density computation to
define define an optimal adversary \citep{balunovic2021fair, cerrato2022fair}. 
%
This allows for obtaining provably fair representations while also obviating the optimisation
challenges that accompany (parametric) adversarial training, though at the cost of an independent
INN for each of the sensitive groups.

\marginpar{\textbf{Practical drawbacks}}
%
As discussed, INNs have a number of unique and compelling properties that would seem to make them
the choice method for many generative purposes; INNs do have their share of practical
shortcomings, however. 
%
Notably, bijectivity does not come at a cost; while there are some ways of mitigating it, such as
factoring out parts of the representation at intermittent steps \citep{hoogeboom2019integer}, one
is constrained to having a latent space that is equidimensional to the input space and when the
latter is large, as in the case of images, training an INN can be computationally challenging.
%
Conventional architectures do not suffer this problem as they can make free use of coarsening
(downsampling) operations throughout their extent.
%
This drawback is further compounded by the fact that the layers making up an INN are necessarily
less expressive than their invertible counterparts and thus more them are needed to achieve
comparable levels of expressiveness in composition. 
%
The coupling layers that constitute the atomic building blocks in \cite{dinh2014nice} restrict
their non-linear, non-invertible functions, to only a subset of the input dimensions at a time so
that the layer as a whole remains invertible, thus limiting the degree to which interdependencies
between the input dimensions can be modelled.
%
Finally, if not properly constrained (e.g. to be bidirectionally \(K\)-Lipschitz), the
optimisation of INNs can be prone to instabilities that can render them \emph{numerically}
non-invertible, despite their design, and thus invalidate computations made according to
Eq.~\ref{eq:cov} \citep{behrmann2021understanding}.
