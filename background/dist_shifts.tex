% ********************************************************************************
\section{A (brief) taxonomy of distribution shifts}\label{sec:dist-shift}
% ********************************************************************************
In this section, I provide a brief taxonomy of the types of distribution shift that arise in the
statistical-learning literature and discuss how and in what contexts they might practically emerge.
%
To this end, I draw heavily upon the works of \citet{moreno2012unifying} and
\citet{castro2020causality} in the definitions, noting that the \ac{ML} literature is not of a
single mind regarding the terminology and its semantics.
%
In \S\ref{sec:lens-of-causality} I will reframe these distribution shifts in causal terms by
introduction of an exogenous variable -- allowing for an elegant formulation of the distribution
shift problem and its relation to invariance -- but I leave that aside for now and seek to present
them in more general terms.
% --------------------------------------------------------------------------------
\subsection{Covariate shift}\label{ssec:covariate-shift}
% --------------------------------------------------------------------------------
As the most well-studied of the shifts, covariate shift refers to a change in the marginal
distribution of the inputs, that is to say we have \( P^{tr}(X) \not\approx P^{te}(X) \) while the
conditional distribution remains (effectively) unchanged, i.e. \( P^{tr}(Y|X) \approx P^{te}(Y|X)
\).
%
% NOTE: still undecided about the below; maybe this (and target shift, conversely) don't have 
% much meaning unless we presuppose the direction of causality.
Departing from \citet{moreno2012unifying}, I do not restrict its definition to problems of a causal
(\( X \to Y\)) nature and do away with the distinction between covariate shift and its anticausal
(\( Y \to X \)) analogue in \emph{prior shift} to simplify exposition.
%
Changes in the distribution of the target variable, \(Y\), will be referred to as \emph{target
shift}, as explained in the subsection below.
%
This is not to say that I disregard the importance of distinguishing between the two causal
directions; in the context of \ac{DA} and \ac{SemiSL}, I will discuss at some length the dependence of these
paradigms on this characteristic of the problem.
%
Indeed, a common assumption in \ac{DA} is that the source and target domains are separated by
covariate shift \citep{david2010impossibility}, however this assumption breaks down when the
problem is anticausal (when we have what \cite{moreno2012unifying} term `prior shift').

% --------------------------------------------------------------------------------
\subsection{Target shift}\label{ssec:label-shift}
% --------------------------------------------------------------------------------
% NOTE: As noted in the note above, we might want to specify that this type of shift only applies
% to anticausal tasks
Diametric to the above, target shift describes, as the name suggests, a shift in the marginal
distribution of the targets, \(Y\), i.e. \( P^{tr}(Y) \not\approx P^{te}(Y) \).
%
In the classification setting, this means that classes do not appear equifrequently in the training
and test data; many real-world datasets used for training exhibit long tails, \wrt{} the classes
(or targets generally), in which the most-frequent class can appear orders-of-magnitude more
frequently than the least-frequent class, while the test data has more even coverage.
%
As discussed in \S\ref{sec:erm}, a classic approach to rectifying this kind of shift, in the case
of the discriminative models we are usually concerned with, is to importance weight the instance
losses or, near-equivalently, the sampling mechanism, using the ratio \(\frac{P^{te}(Y)}{P^{tr}(Y)}
\), or simply by the denominator should \( P^{te}(Y)\) not be reliably estimable (as is often the
case).
%
% This can arise from different predispositions in the training and test populations, or from
% variations in environmental factors.
%
% \subsection{Manifestation shift (conditional).}\label{ssec:manifestation-shift}
% %
% Manifestation shift refers a change in the way prediction targets physically manifest in
% between domains.
% %
% Denoting the random variable corresponding to the domain, or environment, as \(E\), we can then
% couch manifestation shift in mathematical terms as shift as inducing the inequality \( P^{tr}(E |
% Y) \neq P^{te}(E | Y)\).
% %
% As with target shift, this type of shift irremediable unless one employs strong parametric
% assumptions on the nature of the interdomain differences.
% --------------------------------------------------------------------------------
\subsection{Concept shift}\label{ssec:concept-shift}
% --------------------------------------------------------------------------------
To complete the triad of bivariate distribution shifts (I will later revisit distribution shift
under the influence of an exogenous domain or environment variable) we have \emph{concept shift},
referring to changes in the conditional distributions, \( P(Y|X) \) or \( P(X|Y) \), while the
respective (\( P(X) \) and \( P(Y) \)) marginal distributions are preserved.
%
Thus, in the classification setting, concept shift corresponds to a change in the mechanism used to
annotate the data; this might entail, for example, changes in the class definitions, differences in
annotation protocol or grading scales between sites,  or different proclivities/standards in the
annotators in the case of human-driven annotation should the task possess an element of
subjectivity (alignment via \ac{RLHF} being a prime and topical example of such a task
\citep{bai2022training}).
%
In addition to the shifts discussed, one can naturally also consider their composition, giving rise
to \emph{compound shifts}, in which both the marginal and conditional distributions are subject to
change. Such shifts, however, are unusual in the literature, and, perhaps more pertinently,
impossible to solve without invoking strict assumptions due to the need to decouple the constituent
shifts (a problem of identifiability).
%
% \subsection{Acquisition shift (domain shift).}\label{ssec:acquisition-shift}
% %
% Domain or acquisition shift refers to a change in distribution arising due to a change in how or
% where the data was acquired; this commonly stems from differences in measuring
% equipment, such as the type of satellite, in the context of remote sensing, or camera
% trap, in the context of ecology.
% %
% A prime example of such a shift can be found in the medical imaging domain, where differences in
% the scanning equipment used to collect the training and test data can lead to introducing
% site-specific biases that hinder generalisation and cannot be readily rectified by means of
% data-preprocessing \citep{glocker2019machine}.

% --------------------------------------------------------------------------------
\subsection{Sampling bias}\label{ssec:sampling-bias}
% --------------------------------------------------------------------------------
Sampling (which I use synonymously with \emph{selection} and \emph{representational}) bias refers
to distribution shifts that arise due to systematic flaws in the data-collection process that cause
training samples to be selected in a non-uniform fashion from the general population being
modelled.
%
That is to say, the data is not missing at random but rather conditionally, and most notably when
the conditioning is on the target or some other characteristic, such as a particular demographic.
%
Thus, sampling bias is not a type of distribution shift in itself, but rather a mechanism by which
the above-described distribution shifts can emerge, and it is particularly germane to
Chapters~\ref{ch:nifr} and \ref{ch:supmatch} of this thesis in which I consider extreme cases of it
in which certain demographics, or outcomes for certain demographics, are omitted from the training
data, promoting \acp{SC} between said demographics and the outcome.
%
To give an example, in conducting a local survey there will invariably be subsets of the general
population which are under-represented, or altogether excluded, from data-collection due to
availability, willingness, and applicability to the research being conducted; if the locale in
question were a university, then we would expect the population to be significantly younger and
more liberal than on average.
%
Indeed, this problem is particularly well-noted in experimental psychology, in which cohorts
overwhelmingly consist of a very narrow band of individuals from the so-called WEIRD (White,
Educated, Industrialized, Rich, Democratic; \citet{henrich2010weirdest}) group.
%
The experimental data obtained from such homogeneous cohorts has been used by numerous high-profile
journal papers to support broad claims about the general population, despite obvious issues with
its representativity.

%
A prominent yet more subtle, mechanistically, example of sampling bias can be found in the
credit-scoring literature, in which no feedback is obtained from previously rejected candidates;
this leads to bias amplification (as the model's past decisions directly shape the training data at
future iterations) and in the context of fairness, demographic biases incurred due to such feedback
models have been studied under the guises of Delayed Impact \citep{liu2018delayed} and Residual
Unfairness \citep{kallus2018residual}.
%
Indeed, the systematic censoring problem posed by \cite{kallus2018residual} served as a prime
motivator for the setup considered in Chapter~\ref{ch:nifr}, such that in the case of a binary
decision system -- one designed for automated hiring, for instance -- and a population comprised of
two subgroups, only positive outcomes are observed for the advantaged subgroup, while only negative
outcomes are observed for the disadvantaged subgroup.
%
