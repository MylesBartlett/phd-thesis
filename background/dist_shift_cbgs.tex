\begin{figure}[]
        \centering
        % -------- Covariate Shift --------
        \begin{subfigure}[t]{0.3\textwidth}
            \centering
            \caption{\textbf{Covariate}}
            \begin{tikzpicture}[]
                \node[state, gray] (e) [] {$E$};
                \node[state, blue] (x) [right = of e] {$X$};
                \node[state, red] (y) [below = of x] {$Y$};
                % \node[state, red] (x) [] {$X$};

                \path (e) edge (x);
                \path (x) edge (y);
            \end{tikzpicture}
        \end{subfigure}
        % -------- Label Shift --------
        \begin{subfigure}[t]{0.3\textwidth}
            \centering
            \caption{\textbf{Label}}
            \begin{tikzpicture}[]
                \node[state, gray] (e) [] {$E$};
                \node[state, blue] (x) [right = of e] {$X$};
                \node[state, red] (y) [below = of x] {$Y$};
                % \node[state, red] (x) [] {$X$};

                \path (e) edge (y);
                \path (x) edge (y);
            \end{tikzpicture}
        \end{subfigure}
        % -------- Multi-label Shift --------
        \begin{subfigure}[t]{0.3\textwidth}
            \centering
            \caption{\textbf{Multi-label}}
            \begin{tikzpicture}[]
                \node[state, gray] (e) [] {$E$};
                \node[state, blue] (x) [right = of e] {$X$};
                \node[state, red] (y) [below = of x] {$Y$};
                \node[state, magenta] (s) [below = of e] {$S$};

                \path (e) edge (s);
                \path (e) edge (y);
                \path (e) edge (x);
                \path (x) edge (y);

            \end{tikzpicture}
        \end{subfigure}
    \caption{
        Causal Bayesian Graphs (CBGs) corresponding to different distribution shifts, induced by
        exogenous variable, \(E\), for a causal prediction (\(X \to Y \)) task, where \(X\) and
        \(Y\) correspond to the covariates and response variable, respectively.
        %
        (c) introduces an additional, auxiliary label, \(S\), which forms the basis of the problems
        tackled in Chapters 3 an 4.
\end{figure}
