% ********************************************************************************
\section{Supervised learning, empirical risk minimisation, and its pitfalls}\label{sec:erm}
% ********************************************************************************
%
Traditional learning algorithms usually assume  that (or are only optimal when) the training and
test samples are \emph{both} identically-and-independently distributed (\iid{}) random variables,
such that one has \( P^{tr}(X, Y) \approx P^{te}(X, Y) \). 
%
Here, \( P^{tr}(X, Y) \) and \( P^{te}(X, Y) \) denote the (joint) training and test distributions,
respectively.
%
Based on this assumption, the method of (true) risk minimisation seeks the hypothesis \( f^\ast \in
\gF \) that is the minimiser of the eponymous \emph{risk}, $\gR$, defined according to some
statistical distance, or \emph{loss}, \( \gL: \R^\Omega \times \gY^\Omega \to \R \) between the
predicted, \(\hat{Y} \triangleq f(X) \), and ground-truth labels, \(Y\) over the training
distribution, \(P^{tr}(X, Y)\). 
%
A canonical example of such a distance for classifications tasks is the \emph{cross-entropy loss};
in information-theoretic terms, this can be couched as the amount of information required to
identify a sample from the true distribution given a coding scheme optimised for the predictive
distribution and takes the form
%
\begin{equation}
    H(Y, f(X)) \triangleq \E_{P(Y)}[ \log P(f(X)) ].
\end{equation}
%
An alternative, and perhaps more natural, way of viewing this function is by its decomposition into
the sum of the \ac{KL} Divergence (also known as the \emph{relative entropy}),
\(D_{KL}(P(f(X))||P(Y))\), and entropy of the marginal target distribution, \(H(Y)\). 
%
When the latter carries no dependence on the learned parameters -- as is generally the case, save
for certain cases of model-distillation, consistency-regularisation etc.\ -- the term vanishes from
the gradient, leaving just the \acs{KL} term.
%
Returning from this brief aside, we can formally define the (population or true) risk as
%
\begin{equation}\label{eq:pop_risk} 
    %
    \gR (f) \triangleq \E_{ (X, Y) \sim P^{tr}(X, Y) } [ \gL ( f(X, Y) )].
    % \gR(f) \triangleq \E_{ (X, Y) \sim P^{tr}(X, Y) } \[ D (f(X), Y) \].
\end{equation}
%
In practice, of course, one does not have access to the true generative distribution, but only a
finite set of realizations of it that together form a \emph{dataset}, \( \gD^{tr} \), consisting of
observed input-target pairs \( (x, y) \). 
%
Thus, we are restricted to \acl{ERM} (\acs{ERM}; \citealp{vapnik1991principles}), defined as the risk
is instead defined over the \emph{empirical} distribution, a finite set of observations, rather
than over the underlying distribution from which those observations were drawn.
%
\sidepar{\Acl{ERM}}
%
To accommodate this discrepancy, two things need to be accounted for. 
%
First the substitution of \( P^{tr}(X, Y) \) with its empirical counterpart, \( \gD^{tr} \triangleq
\{(x_i, y_i) \}_{i=1}^{N^{tr}} \); since we are now operating over a finite set of \( N^{tr} \)
tuples, the expectation can be replaced with a finite sum (with normalisation). 
%
Second, given the variables are deterministic rather than random, one can no longer frame the
optimisation objective in terms of a statistical distance explicitly. 
%
Instead one measures -- and uses as feedback to drive model-optimisation -- the discrepancy between
the predicted and observed targets using an empirical \emph{loss} function, \( \ell: \R \times \gY
\to \R \). 
%
In standard classification settings in which the targets are given by single (one-hot encoded)
labels, \( Y \) is represented by a degenerate (delta) distribution, wherein for each instance we
simply have an binary indicator of which class said instance belongs to, rather than a distribution
over the probability simplex, allowing for capturing of intrinsic uncertainty in the task.
%
With the foregoing adjustments in mind, one can then define the standard \ac{ERM} objective as
% %
\begin{equation*}\label{eq:emp_risk} 
    %
    \hat{\gR}(f) \triangleq \frac{1}{|\gD^{tr}|}  \sum_{(x, y) \in \gD^{tr}} \ell (f(x), y). 
%
\end{equation*}
%
This version of the objective is simply a uniform (unweighted) average over all training pairs; it
does not take into account the distribution of the inputs or targets.
%
\sidepar{Long-tailedness and importance-weighting}
%
This is mentionable as many real-world datasets exhibit significant class-imbalance
\citep{zhu2014capturing, van2017devil}, or, more generally, `long-tailedness', which is to say that
the marginal distribution \( P(Y) \) is not uniform over its support.
% %
Such motivates replacing the unweighted (or, more accurately, `uniformly weighted') objective given
by Eq.~\ref{eq:emp_risk}, with an \ac{IW} variant wherein the loss is weighted by \( P^{tr}(Y)^{-1}
\), or, in the empirical case, by the inverse frequencies of the targets, in the discrete
(classification) case, or by the empirical density of the target (as given by kernel density
estimation (KDE), for instance) in the continuous (regression) case.
% %
Here, we have assumed no foreknowledge of \( P^{te}(Y) \) -- this typically being the case in
practice -- with the choice of an uninformative, uniform distribution over the domain leading to
its elimination from the \ac{IW} term that in general takes the form \( \frac{ P^{te}(Y) }{
P^{tr}(Y) } \) (or the empirical equivalent).

%
Using \( w \in \R^+ \) to denote the weight assigned to instance \(x\) in \( \gD^{tr} \), we can
then generalise Eq.~\ref{eq:emp_risk} to
% %
\begin{equation*} \label{eq:emp_risk_weighted}
    %
    \hat{\gR}(f) \triangleq \sum_{(x, y) \in \gD^{tr}} w \cdot \ell (f(x), y),
    %
\end{equation*}
%
with `\( \cdot \)' denoting regular scalar multiplication (over \(\R\)), noting that this form
subsumes the unweighted form, which itself can be recovered by simply fixing \(w\) to \(
|\gD^{tr}|^{-1} \) for all instances.
%
It is also worth noting that these weights can be adaptive: they can be iteratively adjusted over
the course of training according to some parametric or non-parametric function
\citep{wang2021importance}.
%
Instead of weighting the instance-losses, one can instead use the weights to adjust the sampling,
which has several practical advantages when training with \ac{SGD}, particularly: 
% %
1) The procedure is non-invasive: no modification to the data-loading nor the computation of the
loss is required;
%
2) Highly-weighted samples appear in batches commensurately often; when weighting the loss, samples
belonging to the tail will appear in batches rarely, resulting in forgetting and poor diversity as
said samples are effectively duplicated.
% %
One can achieve a similar effect by under-sampling the majority classes, groups, or their
intersections, such that they are equifrequent, and \iid{} sampling from that subset \(
\gD^{tr}_{\text{US}} \subset \gD^{tr} \) (or, conversely by duplicating instances from the minority
classes to the same end).
% %
Under- and over-sampling have long been used as a remedy for class imbalance
\citep{chawla2002smote} but the former has recently been shown to be effective -- matching or
exceeding in performance more sophisticated algorithms -- for group-robustness and
spurious-correlation problems \citep{sagawa2020investigation, idrissi2022simple} in part due to its
early-stopping effect.

Yet, despite its long and storied history, with roots in early statistical modelling, the practical
usefulness of importance-weighted \ac{ERM} in the context of modern \acf{DL} has recently been
impugned \citep{byrd2019effect, zhai2022understanding}.
%
\citet{byrd2019effect} demonstrate that for \emph{over-parametrised} models the effects of
importance-weighting diminish over the course of training; these effects can be partially recovered
when used in conjunction with regularisation such as dropout \citep{srivastava14dropout},
early-stopping and standard \(L_2\) weight decay but without such interventions the converged-upon
solution is identical for both \ac{IW}-\ac{ERM} and standard (unaugmented) \ac{ERM}. 
%
Further empirical evidence for this was provided by \citet{sagawa2019distributionally}, wherein the
importance of combining aggressive regularisation with (a dynamic form of) importance-weighting for
strong worst-group generalisation is stressed.
%
Theoretical support for these results was later adduced by \citet{zhai2022understanding} with proofs
showing that equivalence of the implicit biases of these \ac{IW}-based algorithms and \ac{ERM}.
%
Summarily, for all its intuitiveness, importance-weighting, it provably does not alter the solution
to the optimisation problem defined by the training set, which is to say, solutions that attain
zero-loss are invariant under reweighting.
%
This understanding has motivated other approaches, such as those based on polynomially-tailed
losses (for binary classification; \citealp{wang2021importance}) and logit-adjustment
\citep{menon2020long}.
%
% The view of the latter is to instead aim to shift the classification boundary to be closer to the
% dominant (majority) classes and can be realised through the use of per-class margins, either
% through modifying the loss during training or post-hoc correction \cite{fawcett1996combining} 

As statistical models are only required to model correlations in the data to satisfy the loss
function, they ultimately only capture a superficial representation of the true physical processes
involved.
%
\sidepar{The pitfalls of statistical modelling}
%
In the discriminative case (that this thesis is concerned with), for a given \( X \) and \( Y \) we
are interested in approximating the conditional distribution \( P(Y|X) \); this  corresponds to
tasks like predicting the probability that a given image contains a dog (image classification), or
the probability that a given chest X-ray indicates a pulmonary infiltration, or some other thoracic
condition.
%
Indeed, the task of accurately estimating \( P(Y | X) \) can be provably solved by observing a
sufficient amount of \iid{} data drawn from the joint distribution \( P(X,Y) \), yet this only
solves the problem from the aforementioned statistical perspective, and we will see that this
perspective is not always aligned with the causal one, which can lead to problems in generalisation
under certain conditions that arise disconcertingly often in real-world -- many, moreover,
safety-critical -- applications.
%
This is to say, the predictions of a statistical model are only trustable when the conditions of
the training and test distributions are sufficiently similar, and, in short, arbitrary shifts
(interventions on the data-generating distribution) can give rise to arbitrarily bad predictions
\citep{pearl2009causality, scholkopf2012causal}.

The true causal relationships between independent and dependent variables is in general
\emph{unidentifiable} given the training data alone, due to confounding variables; additional
information, as provided by interventions, or \emph{environments} \citep{peters2016causal} -- a
tack popular within the domain generalisation literature, wherein domain can be viewed as a
different intervention on the true distribution -- is needed to resolve the immanent statistical
ambiguity.
%
By `confounding variable', or \emph{confounder}, I mean some variable that that is the causal
parent of two or more other variables and explains the statistical dependency between them despite
those variables not being causally-related themselves; in the trivariate case this corresponds to
the fork \(X \from S \to Y \), wherein there exists a spurious (acausal) correlation between \(X\)
and \(Y\) that is eliminated by conditioning on the confounder \(S\).
%
\sidepar{Short-cuts make for long delays}
%
In the \ac{SCL} problems addressed in Chapters~\ref{ch:nifr} and \ref{ch:supmatch}, we will see
statistical learning break down in a similar way yet for essentially the opposite reason.
%
Namely, instead of having a latent variable that explains the statistical dependency between \(X\)
and \(Y\) in the absence of a causal dependency, we instead have some spurious variable, \(S\) on
which \(Y\) is strongly statistically, but not causally, dependent, with \(X \to Y \) (or some
subset of \(X\)) assumed to be the true causal mechanism.
%
I will broach more deeply what \ac{SCL} is and how the mechanisms that give rise to it in
\S\ref{sec:shortcut-learning};
%
for now, however, we will proceed to characterising some of the types of distribution shift that
statistical learning might contend with.

% % (see \citept{vogel2020weighted} as reference for formulating the traditional ERM setup and that
% of % its weighted counterpart, also see \cite{shimodaira2000improving, wang2021importance, %
% semenova2019study, zhai2022understanding, idrissi2022simple} 

% % ------------------------------------------------------------------------------
% \subsection{Learning under Class Imbalance/Long-tail Learning} %
% ------------------------------------------------------------------------------  

% Notes from and on \cite{menon2020long}: % \begin{itemize} \item Real-world classification
% problems typically exhibit long-tailed label distributions, wherein most labels are associated
% with only a few samples. % \item Owing to this paucity, generalisation on such labels is
% challenging -- a classifier trained on such data following such a distribution is susceptible to
% undesirable bias towards the dominant labels. % This problem has been widely studied in the
% literature on learning under \emph{class imbalance} \citep{cardie1997improving}. % \item Existing
% approaches to coping with class imbalance modify: % \begin{enumerate} \item inputs to the model,
% for example by over- or under-sampling \citep{kubat1997addressing, chawla2002smote} \item outputs
% (logits) of a model, for example by post-hoc correction of the decision threshold
% \citep{fawcett1996combining} % \item internals of a model, e.g. by modifying the loss function
% \end{enumerate}'

% \end{itemize}'


