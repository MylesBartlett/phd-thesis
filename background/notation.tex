% ********************************************************************************
\section{Some notes on notation}\label{sec:notation}
% ********************************************************************************
We describe here some of the general notation schemes used throughout this background chapter,
leave the concrete notation to be defined contextually, both to allow overloading (to allow for
reuse and restrictedness of the alphabet) and to minimise cognitive overhead for the reader.

%
First, we denote random variables using upper-case (non-calligraphic) letters and their associated
observed/deterministic/realised variables with the corresponding lower-case letters.
%
Following convention, we consistently denote by \(X\) and \(Y\) the input (covariate) and
target (response) variables, respectively; by \(S\) some auxiliary variable on which we want to
condition (for evaluation and/or optimisation), such as the domain (in domain
adaptation/generalisation) or sensitive attribute (in algorithmic fairness); by \(Z\) the latent
space, representations, encodings, or embeddings (all synonymously) of some model.
%
Second, calligraphic letters are used to denote (but not exclusively) the domain of a variable,
e.g. \(x \in \gX \).
%
Under this scheme, we would have for the random variable, \(X: \Omega \to \R^d \), realisations \(x
\in \gX \subset \R^d \) defined on a subset of the \(d\)-dimensional space of real numbers.
%
We then use \(P(\cdot)\) to denote probability distributions with conditioning indicated as
\(P(X=x)\) -- continuing the foregoing example -- and use \(\gD\) to denote \emph{datasets} that
correspond to the empirical distributions of variables; for instance \(\gD \triangleq
\{x_i\}_{i=1}^N \) denotes a dataset made up of \(N\) observations of \(X\).
%
We will often augment this notation with super- and subscripts to indicate a variety of concepts
including, inter alia, association with a particular subset of the data or concept, optimality,
observability, and approximation.
%
Some representative examples include \(\gD^{tr}\) and \(\gD^{te}\) to denote the training and test
sets, respectively, \(f^\ast\) to denote the optimal function \wrt{} some optimisation problem, and
\(\hat{y}\) to denote a prediction made by some estimator (of \(P(Y|X)\)).
%
% We will state the exact meaning in each case, whenever it is not obvious by association.

%
Finally, to simplify exposition, we abuse notation by allowing functions of the form \(f: X \to Y
\) to accept random and observed variables interchangeably; we assume that the derived function
classes are Borel Measurable and as such that a function of a random variable is also a random
variable. \(f\) to operate on random variables \(X\).
% %
Thus, pedantically speaking, \( f(X) \) should be read as shorthand for \( f \circ X(\omega) \),
for some event \( \omega \) drawn from sample space, \( \Omega \), while \( f(x) \) should be read
in the standard fashion, with deterministic inputs and outputs.
