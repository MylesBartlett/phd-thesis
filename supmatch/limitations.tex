\subsection{Limitation and intended use}
\label{sec:sm-limitations}
% First, dataset consumers should take extra care about the cost-benefit analysis of selecting particular datasets for their machine learning tasks. 
%
Although having zero labelled examples for some subgroups is not uncommon due to the effects of
systematic bias or dataset curation, we should make a value-judgement on the efficacy of the dataset
with respect to a task.
%
% {\color{red}Corrective action such as the one described in this paper or inaction should be
% recorded.}
We can then decide whether or not to take corrective action as described in this paper.
%
A limitation of the presented approach is that, for constructing the perfect bags used to train the
disentangling algorithm, we have relied on knowing the number of clusters \emph{a priori},
something that, in practice, is perhaps not the case. However, for person-related data, such
information can, for example, be gleaned from recent census data. 
%
(see also Appendix~\ref{sec:sm-overclustering} for results with misspecified numbers of clusters.)
%
% Removing this dependency through automatic determination of the number of clusters would
% generalize our method further but this line of research is beyond the scope of the current paper. 
%
One difficulty with automatic determination of the number of clusters is the need to ensure that
the small
% but salient
clusters are correctly identified. 
%
% In the case of 2-digit Colored MNIST, for example, the deployment set may contain only a small
% portion of {\color{purple}purple} fours relative to the other subgroups, meaning that the cluster
% they form can be easily overlooked by a clustering algorithm in favor of larger but less salient
% clusters (which may be sub-clusters of other digit/color combinations). 
A cluster formed by an underrepresented subgroup can be easily overlooked by a clustering algorithm
in favour of larger but less meaningful clusters.
% less salient clusters (which may be sub-clusters of other, larger, subgroups). 
