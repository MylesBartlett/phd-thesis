% -------------------------------------------------------------------------------
In this section, we illustrate and formalise the problem of classes with incomplete
subgroup-support. 
%
We start by defining requisite notation for conveying our setup and in
\S\ref{ssec:problem_formalism} expand on this notation to construct a more general and compact
description of said problem. 
%
Let \( x \in \gX \subset \R^d \), \( y \in \gY \) and \( s \in \gS \) denote the observed input
features, class labels and subgroup labels, respectively, with \( \gY \) and \( \gS \) being
non-empty, finite sets (i.e. \( \gY, \gS \in \{ A | A \in \gP(V), A \neq \varnothing, |A| <
\aleph_0 \} \)), and with upper-case letters denoting observed variables' random-variable
counterparts here (\(X\), \(Y\), and \(S\)) and throughout.
% the set-builder notation S = {A | A ∈ P(X), A ≠ ∅, |A| < ∞} defines the set of all non-empty,
% finite sets that are subsets of the set X.
%
We refer to the values, \( g \in \gG \) as \emph{sources}, representing unique pairs of \( s \) and
\( y \), such that \(  \gG \subseteq \gS \times \gY \). 
%
As in a standard supervised learning task, we have access to a labelled training set \( \gD^{tr}
\triangleq \{ (x_n, s_n, y_n) \}_{n=1}^{N^{tr}} \subset (\gX \times \gS \times \gY) \), that is used to
train a classifier \( \Gamma:\gX \rightarrow \gY \) that is then deployed on test set \( \gD^{te}
\triangleq \{ (x_n, s_n, y_n) \}_{n=1}^{N^{te}} \subset (\gX \times \gS \times \gY) \). 
%
We use superscript, to denote association of a domain with a correspondingly superscripted dataset,
e.g. \( \gG^{tr} \) and \( \gG^{te} \) respectively denote the sources in the training and test
sets.
%
Lastly, for some functions, we abuse notation and allow the random and observed variables to be
interchanged as inputs; we presuppose such functions (and their domain) are Borel measurable and
thus preserve the type of variable, i.e. a function of a random variable is also a random variable.
%
For example, given function \(f: \gX \to \R \), we may write both \( f(x) \) and \( f(X) \),
meaning by the latter \( f \circ X(\omega) \) for some event \( \omega \in \Omega \).
%
% -------------------------------------------------------------------------------
\subsection{Spurious correlations from missing sources}\label{ssec:walkthrough}
% -------------------------------------------------------------------------------
The spurious correlation (\ac{SC}; \citealp{arjovsky2019invariant}), or shortcut-learning
\citep{valle2018deep, geirhos2020shortcut}, problem is characterised by the presence of some
secondary attribute \(s\) (such as background \citep{beery2018recognition}, texture
\citep{geirhos2018imagenet}, or gender \citep{sagawa2019distributionally, seyyed2020chexclusion}
that confounds the prediction task.
%
We refer to this attribute as the ``subgroup'', in line with algorithmic fairness (\ac{AF};
\citet{barocas-hardt-narayanan}) that is strongly correlated with the target attribute, \(y\), in
the training set, but spuriously so in the sense that the correlation the mapping \( \gS \to \gY \)
is acausal and thus cannot be expected to hold at deployment time. 
%
This correlation is pernicious when \(S\) is of lower complexity (which can be formalised in the
Kolmogorov sense; \citet{scimeca2021shortcut}) than the causal cues contained in \(X\), and thereby
becomes the preferred cue by virtue of simplicity bias \citep{valle2018deep}. 
%
Such problems have garnered considerable attention in recent years \citep{liu2021just,
pezeshki2021gradient, SohDunAngGuetal20, krueger2021out} due to their pervasive, and potentially
catastrophic \citep{codevilla2019exploring, de2019causal, castro2020causality}, nature.
%
In this paper, we introduce, and propose a semi-supervised solution for, a hierarchical and
class-asymmetric variant of the \ac{SC} problem that we term the \emph{\acf{MS}} problem .

To illustrate the general \ac{SC} problem and the \ac{MS} problem as a particular instantiation of
it, we define the conditional-probability matrix, \( \mathbf{P}^{tr} \in [0, 1]^{|\gS| \times
|\gY|} \), where each element \( \mathbf{P}^{tr}_{ij} \) encodes the conditional probability \(
P^{tr}(Y=j|S=i) \) in the training set, \( \gD^{tr} \). 
%
When \( \mathbf{P}^{tr} \) is both binary and doubly stochastic (that is, has all rows and columns
summing to 1) we have that \(y\) is completely determined by \(s\) in \( \gD^{tr} \) -- this is an
extreme form of the \ac{SC} problem which is statistically intractable without access to additional
sources of data \citep{KehBarThoQua20} or multiple environments \citep{arjovsky2019invariant}. 
%
The \ac{MS} problem can be viewed as a relaxation of this \ac{SC} wherein the elements of \(
\mathbf{P}^{tr} \) respect the constraint that all columns contain at least one non-zero value,
i.e. we observe all class labels but not all possible pairs of class and subgroup labels -- we say
that we have \emph{missing sources}, \(\gM\ \triangleq \gG^{te} \setminus \gG^{tr}\). 
%
This setup still leads to spurious correlations but ones that are statistically tractable due to
asymmetry.
%
Practically speaking, considering only cases where sources are entirely missing is overly
restrictive, and as such we instead view the problem setup as extending to cases where sources may
not be altogether missing but have sample sizes too small to constitute meaningful supervision. 
%
To understand the non-triviality of this problem, and why aiming for invariance to \(s\) in the
training set alone -- as is characteristic of many representation-learning methods in \ac{AF}
\citep{edwards2015censoring, madras2018learning, quadrianto2019discovering}  and domain adaptation
(\ac{DA}; \citep{ganin2016domain, zhao2018adversarial, saito2018maximum, lee2019sliced}) -- will
assuredly fail, consider a binary classification problem with binary subgroups, where \(\gY = \{ 0,
1 \}\) and \( \gS = \{ 0, 1 \}\) and for which \( \mathbf{P}^{tr} \) takes the form
%
\begin{align}\label{ms_example}
  \mathbf{P}^{tr} = \bordermatrix{
  & Y=0 & Y=1 \cr
  S=0 & 0.5 & 0.5 \cr
  S=1 & 1.0 & 0.0}~.
\end{align}
%
This represents a special case of the \ac{MS} problem that we refer to as the \emph{Subgroup Bias}
(SB) problem, distinguished by the fact that we observe all subgroups. 
%
Here, we have samples from $S=0$ evenly distributed across both the negative and positive classes;
for $Y=1$, however, we only observe samples from the negative class. 
%
This setup might appear somewhat benign at first blush, given that all classes are present in the
training set, however, the fact that \(s\) serves as a proxy for \(y\) in the case of $S=1$
frustrates our goal of subgroup-invariant classification. 
%
The reason for this becomes obvious when decompose a classifier into a mixture of experts (MoE),
where \(s\) indicates which expert to choose for the given sample. 
%
Such a model naturally arises in practice due to the tendency of deep neural networks to strongly
favour shortcut solutions \citep{geirhos2020shortcut}.
%
We note that for this, and throughout the paper, we assume that \(s\) is inferable, to some extent,
from $x$, that is \( \gI(X; S) > 0 \), with \( \gI(\cdot; \cdot) \) denoting the mutual information
between two variables -- this is almost always the case in practice but we make the dependence
explicit here by denoting by \( X_Y \) the causally-relevant component of \( X \)), that is
independent of \(S\), and by including \( s \in \{0, 1\} \) explicitly in the set of inputs.
%
With this noted, we may then define the MoE classifier, \(c_{MoE}\), that `solves' the training set
with labels distributed according to \( \mathbf{P}^{tr} \) as
%
\begin{align}
  c_{MoE}(X_Y, S) = \begin{cases}
c_{S=0}(X_Y) &S=0 \\
0 &S=1 \\
\end{cases},
\end{align}
%
using \(c_{S=0}(\cdot)\) to denote the expert that learns to classify only the subset of the data
for which $S=0$.
%
Such a classifier is clearly undesirable, as should it ever encounter a sample belonging to
subgroup $1$ with a positive label, the classifier will automatically declare it negative without
needing to attend to $X_Y$ -- it is invariant to $X_Y$ while being variant to \(S\), which is the
opposite of what we desire. 
%
This is often done by learning an encoder $f: \gX \to \gZ$ that maps an input $x$ into a
representation, \( z \in \gZ \subset \R^l \), which has the desired property of \(S\)-invariance,
\( Z \perp S \), while also maximising $\gI(Z; Y)$ so that the representation is useful for
classification. 
%
A popular way of imparting this invariance is with adversarial methods \citep{ganin2016domain,
zhao2018adversarial, madras2018learning} where  $f$ is trained to the equilibrium point, $f^\ast$,
of the (non-convex) minimax equation 
%
\begin{align}\label{eq:moo}
\underset{f \in \gF}{\text{min}}\; \underset{a \in \gA}{\text{max}}\,
\E_{(x, s, y) \sim \gD^{tr}}
\big[ 
  \textcolor{red}{ \overbrace{ a(f(x))_s }^{ \text{invariance}} }
  - \textcolor{blue}{ \underbrace{ \lambda \gI(f(x); y) }_{ \text{classification} } }
\big],
\end{align}
%
where \( a: \gZ \to \bigtriangleup^{|\gS|} \) is a parametric adversary with codomain the standard
simplex over \( \gS \), and \(\lambda \in \R^+\) is a positive scalar controlling the trade-off
between the two constituent objectives. 
%
Under ideal conditions, when all possible pairs of \(s\) and \(y\) are observed, \(f^\ast\)
corresponds to the point at which \(a\) is maximally entropic and occurs when \(Z\) is invariant to
\(S\), and only \(S\), while mutual information \wrt{} \(Y\) is jointly maximised -- from an
optimisation standpoint, the gradients of first and second objectives are non-conflicting (i.e.
have non-negative inner products; \citealp{yu2020gradient}) and there is no trade-off. 
%
However, in cases where we have missing sources, the waters are muddied: satisfying the first part
of the objective connotes invariance not only to \(S\), but also to \(Y\), since \(S\) can be
predicted from \(Y\) with above-random accuracy due to the skewed statistics of the dataset. 
%
This is patently problematic as \(Y\) is the very thing we wish to predict and achieving invariance
to \(S\) does little good if our classifier can no longer utilise features predictive of \(Y\). 

Since we cannot achieve optimality for the competing invariance and classification terms
simultaneously, we instead have a set of \ac{PO} solutions that collectively make up the Pareto
front -- learning the solutions corresponding to different trade-offs, or preference vectors, is
the domain of \acl{MOO} (\acs{MOO}; \citealp{deb2014multi}). 
%
Specifically, Eq.~\ref{eq:moo}, with \(\lambda\) controlling the preference direction,
characterises the most straightforward approach to MOO, called \emph{linear scalarisation}
\citep{boyd2004convex}. 
%
\Ac{MOO} has recently been explored in the context of \ac{UDA}, for controlling the descent
direction, in unsupervised domain adaptation (UDA) in light of the conflict arising between the
gradients of the alignment and classification terms \citep{liang2021pareto}, and in \acf{AF} for
controlling the inherent trade-off between predictive performance and fairness (typified by the
\emph{Accuracy-Fairness trade-off}; \citet{martinez2020minimax}).
%
While our missing-sources problem admits a \acs{MOO}-based approach, we are instead interested in
leveraging unlabelled data to sidestep the implied trade-off altogether.
% -------------------------------------------------------------------------------
\subsection{Formalising the problem}\label{ssec:problem_formalism}
% -------------------------------------------------------------------------------
In order to provide a general formulation of the \ac{MS} problem exemplified above, we begin by
defining additional notation for reasoning over label-conditioned subsets and their support. 
%
For a given dataset, \( \gD \), we denote by \( \gD_{S=s^\prime} \) its subset with subgroup label
\( s^\prime \in \gS \), by \( \gD_{Y=y^\prime} \) its subset with class label \( y^\prime \in \gY
\), and -- combining the two -- by \( \gD_{S=s^\prime,Y=y^\prime} \) its subset with subgroup label
\( s^\prime \) and class label \( y^\prime \).
%
According to this scheme, \( \gD_{S=\text{\color{purple}{purple}}, Y=\text{2}} \) should then be
read as ``the set of all samples in \( \gD \) with class label `2' and subgroup label
`\textcolor{purple}{purple}'''.
%
We apply similar syntax to the subgroups, writing \( \gS^{tr}_{Y=y^\prime} \) to mean the observed
subgroups within class \(y\) in the training set.
%
For instance, \(\gS^{tr}_{Y=1}=\{0\}\) prescribes that for class \(1\), only subgroup \(0\) is
present in the training set.

We assume a problem of a hierarchical nature.
%, as illustrated in Fig.~\ref{fig:problem-setup}. 
%
While the full set of class labels is observed in both the training and test sets, we do not
observe all pairs of \(s\) and \(y\) in the former, i.e. \( \gG^{tr} \subset \gG^{te} \) or \(\gM
\neq \varnothing \). 
%
Equivalently, we say that for some class, \(y^\dagger\), we have \( \gS^{tr}_{Y=y^\dagger} \subset
\gS \), subject to the constraint that \(  \gS^{tr} = \gS^{te} \).
%
With this, we can succinctly notate the SB problem realised by Eq.~\ref{ms_example}, in which class
\(Y=1\) has no overlap with subgroup \(S=1\), as \(\gS^{tr}_{Y=1}=\{0\}\) (while
\(\gS^{tr}_{Y=0}=\{ 0, 1 \}\)), corresponding to \( \gM = \{(1, 1)\} \), and distinguish SB
problems generally by the inclusion of the additional constraint \( \gS^{tr} = \gS^{te} \).
%
To illustrate a more complex case, the SB problem depicted in Fig.~\ref{fig:problem-setup}, in
which for we observe exclusively purple `2's and green and red `4's, can be notated with the pair
\( \gS^{tr}_{Y=2}=\{\text{\textcolor{purple}{purple}\}} \), \( \gS^{tr}_{Y=4}=\{
\text{\textcolor{green}{green}}, \text{\textcolor{red}{red}} \} \).
% -------------------------------------------------------------------------------
\subsection{A way forward}
% -------------------------------------------------------------------------------
In this paper, we propose to alleviate the SB problem by mixing labelled data with
\emph{unlabelled} data that is usually much cheaper to obtain \citep{ChaSchZie06}, referring to
this set of \emph{unlabelled} data as the \emph{deployment set} 
%
\footnote{In our experiments, we report accuracy and bias metrics on another independent test set
instead of on the unlabelled data that is available at training time.} \( \gD^{dep}_\star =\{(x_n,
s_n^\star, y_n^\star)\}_{n=1}^{N^{dep}} \subset (\gX \times \gS \times \gY) \), using ``\(\star\)''
to denote that the labels are \emph{unobserved}, and in practice we only have access to \(
\gD^{dep} \triangleq \{(x_n)\}_{n=1}^{N^{dep}} \subseteq \gX \) and must estimate the corresponding
sources.
%
We assume that this deployment set is source-complete \wrt{} the test set, \( \gG^{dep} = \gG^{te}
\). Leveraging this deployment set, we seek to learn a classifier, \( \Gamma \), that can
generalise well to the missing sources appearing in the test set without seeing any labelled
representatives in the training set. 
%
In practice, we treat \( \Gamma \) as a composition, \( c \circ f \), of two subfunctions: an
encoder \( f: \gX \to \gZ \), which maps a given input \( x \) to a representation \( z \in \gZ
\subseteq \R^l \), and a classifier head \( c: \gZ \to \gY \) which completes the mapping to the
space of class labels, \( \gY \). 
%
Since the task of achieving independence between the predictions and subgroup labels can be reduced
to the task of learning the invariance \( Z \perp S \); we next discuss how one can learn an
encoder satisfying this condition in a theoretically-principled manner.
% -------------------------------------------------------------------------------
% % ********************************************************************************
\section{Some notes on notation}\label{sec:notation}
% ********************************************************************************
We describe here some of the general notation schemes used throughout this background chapter,
leave the concrete notation to be defined contextually, both to allow overloading (to allow for
reuse and restrictedness of the alphabet) and to minimise cognitive overhead for the reader.

%
First, we denote random variables using upper-case (non-calligraphic) letters and their associated
observed/deterministic/realised variables with the corresponding lower-case letters.
%
Following convention, we consistently denote by \(X\) and \(Y\) the input (covariate) and
target (response) variables, respectively; by \(S\) some auxiliary variable on which we want to
condition (for evaluation and/or optimisation), such as the domain (in domain
adaptation/generalisation) or sensitive attribute (in algorithmic fairness); by \(Z\) the latent
space, representations, encodings, or embeddings (all synonymously) of some model.
%
Second, calligraphic letters are used to denote (but not exclusively) the domain of a variable,
e.g. \(x \in \gX \).
%
Under this scheme, we would have for the random variable, \(X: \Omega \to \R^d \), realisations \(x
\in \gX \subset \R^d \) defined on a subset of the \(d\)-dimensional space of real numbers.
%
We then use \(P(\cdot)\) to denote probability distributions with conditioning indicated as
\(P(X=x)\) -- continuing the foregoing example -- and use \(\gD\) to denote \emph{datasets} that
correspond to the empirical distributions of variables; for instance \(\gD \triangleq
\{x_i\}_{i=1}^N \) denotes a dataset made up of \(N\) observations of \(X\).
%
We will often augment this notation with super- and subscripts to indicate a variety of concepts
including, inter alia, association with a particular subset of the data or concept, optimality,
observability, and approximation.
%
Some representative examples include \(\gD^{tr}\) and \(\gD^{te}\) to denote the training and test
sets, respectively, \(f^\ast\) to denote the optimal function \wrt{} some optimisation problem, and
\(\hat{y}\) to denote a prediction made by some estimator (of \(P(Y|X)\)).
%
% We will state the exact meaning in each case, whenever it is not obvious by association.

%
Finally, to simplify exposition, we abuse notation by allowing functions of the form \(f: X \to Y
\) to accept random and observed variables interchangeably; we assume that the derived function
classes are Borel Measurable and as such that a function of a random variable is also a random
variable. \(f\) to operate on random variables \(X\).
% %
Thus, pedantically speaking, \( f(X) \) should be read as shorthand for \( f \circ X(\omega) \),
for some event \( \omega \) drawn from sample space, \( \Omega \), while \( f(x) \) should be read
in the standard fashion, with deterministic inputs and outputs.

