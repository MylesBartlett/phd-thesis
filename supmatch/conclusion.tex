%
\corr{
  %
In this paper, we formalised the problem that systematic bias or dataset curation resulting in one or more
subgroups having zero labelled data; we hope our doing so stimulates serious consideration
for it in the planning, building, and evaluating systems.
%
This complements concurrent research by \citet{yang2023change} wherein the same problem (with
different formalism) is alluded to, but its solution left as an open question.
%
Contrastingly, we proposed here a two-step approach for addressing the problem within a
semi-supervised framework. 
%
The first step entails constructing hierarchically-balanced bags from an unlabelled deployment set
via one's semi-supervised clustering algorithm of choice.
%
The second step then entails matching the support, instead of the raw distributions, of the
training and deployment sets in representation space, so as to learn subgroup-invariant
representations, an outcome we prove corresponds to the optimum of the proposed objective function.
%
We empirically validate our frame\-work on the Coloured MNIST and CelebA datasets, showing it
possible to maintain high performance on subgroups with incomplete support.
%
Furthermore, we bound the error in the objective incurred due to imperfect clustering and show that
the proposed set-wise discriminator is empirically more robust to this error than conventional
instance-wise discriminators.
%
Future work includes the exploration of other \ac{UL} methods for realising the \ac{MI} component
of the objective and addressing the limitations raised in \S\ref{sec:sm-limitations}.

\subsection{Dataset representativity}
%
The results presented in this version of the paper are for toy (Coloured MNIST) or quasi-toy
(CelebA) datasets; though both datasets have featured extensively in the literature on
distributional-robust learning (\citealp{arjovsky2019invariant, kim2019learning,
sagawa2019distributionally, creager2021environment}, inter alia) it is natural to doubt the
representativeness of results on them in relation to real-world problems, particularly so as
sub-sampling is required to align them with our \ac{MS} problem.
%
%
To attempt to shore up this shortcoming (if only partially), we have since conducted experiments
with two datasets more emblematic of real-world realisations of the proposed problem, namely
Chest-Xray8 \citep{wang2017chestx} and NICO++ \citep{zhang2023nicopp}.
%
The former dataset is appealing as it coincides with the motivating example proffered in
\S\ref{sec:sm-intro}; the latter is so as its class-subgroup structure allows for demonstration in
a non-binary (\wrt{} both marginals) setting, whereas the results contained herein are
preponderantly binary out of both simplicity and convention.
%

\subsection{Sensitivity analysis \wrt{} bag-balancing}
%
As noted, we provided bounds on the error in alignment propagated by the error in
approximating \( \gD^{dep}\) but empirical analysis of the reification of this is lacking. 
%
Accordingly, we have sought to effect this via controlled sensitivity analyses \wrt{} the
balancing.
To achieve this, we run our support-matching algorithm with ground-truth labels used for balancing
but with different proportions (typically \( \{5\%, 10\%, \dots, 50\%\} \)) of said labels
perturbed (flipped or randomly shifted cyclically). 
%
Rather than randomly (uniformly) perturbing the labels by sampling from the set of complementary
labels -- i.e., \( \tilde{g}_i \sim \mathrm{uniform}(\gG \setminus \{g_i\}) \), with \( g_i \) and
\( \tilde{g}_i \) denoting the ground-truth and the perturbed labels, respectively -- which yields
unrealistic perturbations due to the discounting of semantic relationships between groups
(semantically similar groups are more likely to be confused by a clustering algorithm) we instead
sample a perturbed label, $\tilde{g}_i$ from $\gG$ with probability proportional to the similarity
between the featurisation of the associated sample and the \(\tilde{g}_i\)th centroid, \(
\phi_{\tilde{g}_i} \in \mathbb{R}^d \), with the constraint that the perturbed label does not equal
the original one. 

%
The featurisation is performed using a pre-trained (via contrastive captioning) CLIP
\citep{radford2021learning} visual encoder. 
%
Namely, the centroids \( \Phi \in \mathbb{R}^{|\gG| \times d} \) are computed as the
group-conditional means of the features, over the deployment set, and their similarity with a
given sample's features is measured using cosine similarity. 
%
Assuming \(L_2\)-normalised CLIP features, $\bar z_i^\text{CLIP} \in \mathbb{R}^d$, and prototypes,
$\bar \Phi$, the sampling scheme used to generate the perturbed label, $\tilde{g}_i$, for a given
ground-truth label, $g_i$, can be written as
%
\begin{align}
  %
  \tilde{g}_i \sim \text{Cat}(\gG, \bigtriangleup_\mathbf{w} ), \quad
  %
  \bigtriangleup_\mathbf{w} \triangleq  \frac{\mathbf{w}}{\sum_j \mathbf{w}_j} , \quad
  %
  \mathbf{w} \triangleq \text{exp}(\bar z_i^\text{CLIP} \cdot \bar \Phi \tau^{-1})
  \odot (1 - e_{g_i}),
  %
\end{align}
%
where \( \text{Cat} \) denotes the categorical distribution with support \(\gG\) and sampling
probabilities \( \bigtriangleup_w \), \( \tau \in \mathbb{R}^+_\ast \) denotes a temperature
parameter modulating the sharpness of the sampling distribution, \(e_{g_i}\) denotes the one-hot
encoding of \(g_i\), and \(\odot\) denotes the Hadamard product that is used with \(e_{g_i}\) to
mask out the \(g_i\)th prototype and thereby enforce \( \tilde{g}_i \neq g_i \).
%
% CORRECTED: Include any work that you have done since.
}

