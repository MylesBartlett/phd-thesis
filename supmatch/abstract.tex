\noindent
%
When trained on diverse labelled data, machine learning models have proven themselves to be a
powerful tool in all facets of society.
%
However, due to budget limitations, deliberate or non-deliberate censorship, and other problems
during data collection and curation, the labelled training set might exhibit a systematic dearth of
data for certain groups. This problem is particularly pertinent in medical imaging where the number
of positive samples typically outweigh the number of negative samples by an orders of magnitude and
certain demographics may be excluded on safety grounds (e.g. pregnant women) or due to
socioeconomic biases.
%
We investigate a scenario in which the absence of certain data is linked to the second level of a
two-level hierarchy in the data.
%
Inspired by the idea of protected groups from algorithmic fairness, we refer to the partitions
carved by this second level as ``subgroups''; we refer to combinations of subgroups and classes, or
leaf nodes in aforementioned hierarchy, as \emph{sources}.
%
To characterize the problem, we introduce the concept of classes with \emph{incomplete subgroup
support}. The representational bias in the training set can give rise to spurious correlations
between the classes and the subgroups which cause standard classification models to generalize
poorly to unseen sources.
%
To overcome this bias, we make use of an additional, diverse but unlabelled dataset, called the
\emph{deployment set}, to learn a representation that is invariant to subgroup. This is done by
adversarially matching the support of the training and deployment sets in representation space
using a set discriminator operating on sets, or \emph{bags}, of samples.
% that uses a set-based loss function inspired by multiple instance learning.
%
In order to learn the desired invariance, it is paramount that the bags are balanced by class; this
is easily achieved for the training set, but requires using semi-supervised clustering for the
deployment set.
%
We demonstrate the effectiveness of our method on several datasets and
realisations of the problem.
