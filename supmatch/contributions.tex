\noindent\textsc{Contributions:}
%
\begin{itemize}
    %
    \item 
    %
        Conceptually, I proposed we convert the problem of distribution-matching, proposed by T.
        Kehrenberg, into one of support-matching by means of source-balanced bags and a
        set-discriminator, an necessary element for achieving the desired goal of
        subgroup-invariance while preserving variance to the target.
    %
        Practically, while much of the original codebase was written by T. Kehrenberg, the lion's
        share of the (several-times) rewritten and extended (additional datasets, baselines,
        discriminator methods, etc.) one was authored by me; a similar story applies both to the
        text, with much of the latest version (save the theoretical sections) being of my hand, and
        to the experiment-running (and the implied model-selection).

    \item 
      %
        T. Kehrenberg conceived of the initial idea of overcoming the limitation of the
        partially-annotated representative set in Chapter 3 through the use of distribution
        matching, wrote much of the original implementation and text, and was the primary
        experiment-runner during the nascent distribution-matching stage of the project.
    %
        Later on in the project, he notably worked to establish theoretical guarantees
        for the support-matching method and a more rigorous formulism of the problem setup, while
        also continuing to aid with experiment-running and paper-writing (though both to a reduced,
        but still significant, degree).
    %
    \item
      %
        V. Sharmanska conceived, and gave the first formulation of, the problem setting, wrote
        significant portions of the initial versions of the paper -- those related to the
        introduction and problem setup primarily.
      %
        In later stages of the project, she took on an important advisory role and gave
        feedback on revisions of the paper.
    %
    \item 
      %
        N. Quadrianto suggested the combining of distribution-matching with clustering-derived
        sample-weighting during the initial stages -- this being a major stepping stone in the
        development of the eventual support-matching method (for which said weighting was replaced
        by exact bag-based balancing) -- wrote significant portions of the original text, and ran
        experiments primarily on the clustering side.
      %
        He was also responsible for generally supervising the project, by which I mean discussing
        and advising on current progress and future avenues, and providing feedback on revisions of
        the paper, to give a non-exhaustive list.
      
    %
\end{itemize}
