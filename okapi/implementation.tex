\paragraph{Data Augmentation} We follow \cite{SagWeiLeeGaoetal22} when defining the augmentations
for the the WILDS datasets.
%
In the case of PovertyMap-WILDS we corroborate the original finding that data-augmentation
adversely affects performance, and, in light of this, elect only to use data-augmentation for
FixMatch where it is needed to generate the weak and strong views used in computing the consistency
loss.
%
Since Okapi uses an NN-based approach for generating these views, it is decoupled from the
augmentation strategy and problems that can arise from its misspecification.

\paragraph{Architecture} For our image experiments, contrary to \cite{SagWeiLeeGaoetal22}, we opt
to use the recently proposed ConvNeXt architecture \citep{liu2022convnet}, finding this change to
provide large performance gains and to be crucial in enabling semi-supervised methods to surpass
the ERM baseline.
%
This is in line with \cite{kim2022broad} who similarly found that a change of architecture
(combined with large-scale pre-training) could greatly bolster performance on the iWildCam dataset.
%
More precisely, we use the \emph{tiny} variant of ConvNeXt, pre-trained on ImageNet 1k, as the
initial backbone for our models. We compose this with a single fully-connected layer to construct
the complete predictor both for the target and the propensity score. For our CivilComments
experiments, in contrast, we do not diverge from \cite{SagWeiLeeGaoetal22} in our choice of
architecture, with all models trained with a pre-trained DistilBERT \citep{sanh2019distilbert}.

\paragraph{Optimisation} For optimising all models, we use the AdamW optimiser
\citep{loshchilov2018decoupled} coupled with a cosine annealing schedule without warm restarts
\citep{DBLP:conf/iclr/LoshchilovH17}.
%
We set the initial learning to be $1 \times 10^{-4}$ across the board, and forgo the use of weight
decay. 
%
Models are trained for $120$K, $30$K, and $20$K iterations for iWildCam, PovertyMap, and
CivilComments, respectively.
%
The decay coefficient, $\zeta$, for the target encoder's exponential moving average is initialised
to $\zeta_{\mr{start}}$ and is linearly increased to $\zeta_{\mr{end}}$ over the course of
training. For PovertyMap we set $\zeta_{\mr{start}}$ and $\zeta_{\mr{end}}$ to be $0.996$ and
$1.0$, respectively; for iWildCam, we set $\zeta_{\mr{start}}$ and $\zeta_{\mr{end}}$ to be $0.999$
and $0.999$, respectively, resulting in a fixed value of $\zeta$; $1.0$, respectively; for
CivilComments, we set $\zeta_{\mr{start}}$ and $\zeta_{\mr{end}}$ to be $0.996$ and $0.996$,
respectively, again resulting in a fixed value of $\zeta$. 
 %
We similarly warm up the pre-factor for the consistency loss, $\lambda$, according to a linear
schedule during the first $10\%$ of training to allow a period for the encoder to learn meaningful
relations between samples through the supervised loss before bootstrapping with the consistency
loss, with a final value of $1$.

\paragraph{Matching}\label{matching_imp} In order to determine suitable hyperparameters, $\xi$, for
\CNN, we perform a grid-search in the static setting, using a fixed model. Specifically, we use the
backbone of an ERM-trained model as the encoder with which to generate the queries and keys for
matching.
%
The quality of matching with a given instantiation of $\xi$ is measured using two metrics commonly
used in the statistical matching literature: \emph{Variance Ratio} (VR) and \emph{Standard Mean
Differences} (SMD) \citep{rubin2001using}. 
%
Both metrics operate on pairs of domains, but can be generalised to work when $s$ is non-binary by
simply aggregating over all pairwise results.
%
For a given pair of domains, VR is defined as the ratio of the variances of the covariates between
the two domains, with an ideal value of $1$, while SMD is defined as the difference in their
covariate means, normalised by the standard deviation for each covariate, and is to be minimised.
%
While our proposed method is applicable whether $s$ is binary or categorical, for the experiments
in this paper we take advantage of the fact that the WILDS datasets specify splits with
non-overlapping domains and match from \( \gD_l \rightarrow \gD_u \) and in the reverse direction (from
\( \gD_u \rightarrow \gD_l \)). This decision was based on preliminary experiments which found the binary
variant generally enjoyed more stable optimisation, something which future work should seek to
rectify.
%
In the case of PovertyMap, however, the training splits themselves do not satisfy the
aforementioned requirement of being sourced from mutually exclusive sets of domains and we instead
treat the OOD validation set as \( \gD_u \) (and treat it as being unlabelled w.r.t. $y$, in that it is
only used for \( \gL_\mr{unsup} \)).

