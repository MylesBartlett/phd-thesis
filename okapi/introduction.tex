\section{Introduction}\label{sec:intro}
% ------------------------------------- 
Machine learning models have been deployed for safety-critical applications such as disease
diagnosis \citep{watson2019clinical} and self-driving cars \citep{yu2020bdd100k}, and in socially
important contexts such as the allocation of healthcare, education, and credit (e.g.
\cite{DunYiLanReetal19, HurAde17}). 
%
% Efficiency can be improved, costs can be reduced, and personalisation of services and products
% can be greatly enhanced -- these are some of the drivers for the rapid development and deployment
% of machine learning algorithms. 
%
Many machine learning algorithms, however, rely on supervision from a large amount of labelled
data, and are typically trained to exploit complex relationships and distant correlations present
in the training dataset. 
%
This strategy has proven to be effective in the setting when we have training (source) and test
(target) data that are i.i.d.

In reality, machine learning models are often deployed on target data whose distribution is
different from the source distribution they were trained on. 
%
For example, in the task of classifying animal species in a camera trap image, one aims to learn a
model that can generalise to new camera trap locations despite variations in illumination,
background, and label frequencies, given training examples from a limited set of camera trap
locations.
%
Exploiting correlations that only hold in these limited locations but not in the new locations can
hurt out-of-distribution (OOD) generalisation.
%
While we only have a small subset of camera traps that have their images labelled, we have a large
amount of unlabelled data from the other camera traps that capture diverse operating conditions. 
%
In general, unlabelled data is much more readily available than labelled data and can often be
obtained from distributions beyond the source distribution.
%
Taking advantage of these unlabelled data during training is a key element to build robust models
that have good OOD performance without sacrificing in-distribution (ID) performance.
%
%It therefore seems natural to exploit as many data from different domains as possible, even if
%that data is partially or completely unlabelled. 

Our work is a direct response to the empirical conclusions of \cite{SagWeiLeeGaoetal22} for the
WILDS 2.0 dataset, which extends the WILDS benchmark datasets of \cite{koh2021wilds} through the
addition of unlabelled data. %
In \citet{SagWeiLeeGaoetal22} a variety of state-of-the-art methods leveraging unlabelled data, including
domain-invariant, self-training, and self-supervised methods were evaluated for their ability to
improve OOD generalisation.
In the all but a few cases, however, these methods failed to outperform the combination of effective
data-augmentation and standard empirical risk minimisation (ERM), and among those select cases
none persisted across datasets.

We show that it is possible to make effective use of large volumes of unlabelled data
as supplement to a smaller set of labelled data, from a limited set of domains, to achieve
strong generalisation to data from domains outside the training distribution.
% Given full access to the class labels, one can consider the causal matching approach of
% \cite{mahajan2021domain} boasts both intuitiveness and rigorous grounding in causal theory. 
% In the absence of object identifiers, MatchDG dictates matching pairs of samples
% belonging to the same class that are close in representation space to serve as inputs to a
% contrastive loss.
% In the absence of class labels, however, as in the semi-supervised regime posed by the WILDS 2.0
% dataset, such an approach does not hold and matching unconditionally without constraint can yield
% poor matches, the detriment of which is greatly amplified when the matches are obtained via, and in
% turn used for, bootstrapping.
% Large unlabelled data offer diverse operating conditions but might dilute the information
% We need to find a way to identify and link labelled and unlabelled data that correspond to similar
% statistical units but that need not be the same real-world entity.
%
We turn to a statistical matching (SM) framework \cite{RomInsShaQua22,rosenbaum1985constructing, rubin1973matching}, a model-based approach for providing joint information on variables and
indicators collected through multiple sources. 
%
SM has been widely utilised to assess the effect of interventions in numerous fields, such as
education, medical and community policies (e.g. \cite{biglan2000value, christian2010prenatal}).  
%
In SM, intervened units are paired with control units and those units without a sufficiently-good
match according to a given statistical criteria are excluded when estimating the treatment effect.
%
In the running example of animal-species classification, intervened units may correspond to the
limited set of camera trap locations that are fully-annotated, while control units refer to the
many more camera trap locations that are only partially annotated.
%
Pairing is beneficial for capturing diverse operating conditions, yet the ability to drop unpaired
units is crucial for mitigating the risk of statistically-poor matches corrupting the training
signal.

By developing an online method for statistically matching samples from different domains
(camera-trap locations) and using this to define a consistency loss, we arrive at our proposed
semi-supervised method, \emph{Okapi}. 
%
This consistency loss is predicated on the simple idea of pulling together similar samples from
different domains within the latent space of the encoder, and using this to bootstrap said encoder
such that the distributions become progressively more aligned over the course of training. 
%
Since matching samples using the full dataset at each step of training is computationally
infeasible, we instead approximate it using a combination of momentum-encoding and a memory-bank
that has been well-proven in self-supervised learning \citep{he2020momentum, koohpayegani2021mean}.
%
Compared with other consistency-based methods such as FixMatch \citep{sohn2020fixmatch}, Okapi has
the advantage of being agnostic to both the task and the modality, in addition to being
distributionally robust.
%
Contrary, to \citet{SagWeiLeeGaoetal22}, we show that the supplementary unlabelled data and domain
information can be leveraged by Okapi to improve upon standard ERM on datasets from the WILDS 2.0
benchmark.
