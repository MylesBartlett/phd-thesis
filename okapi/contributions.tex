%
{\renewcommand\labelitemi{}
%
\begin{itemize}
    %
  \item 
    %
    \textbf{I} conceived of the overall Okapi framework, performed a preliminary literature review,
    wrote the core code for it and the baseline methods, configured and ran all
    non-CaliperNN-specific experiments, and wrote the lion's share of the paper.
    %
  \item 
    %
    \textbf{S. Romiti} was responsible for developing the CaliperNN algorithm -- as published in
    \citet{RomInsShaQua22} -- central to the Okapi algorithm, and naturally for the writing of its
    reference code (adapted for the current work for generality/scalability/integrability).
    %
    Within the scope of said work itself, S. Romiti performed, and analysed results of
    (quantitatively and visually), experiments with the aforementioned CaliperNN algorithm,  helped
    analyse the results of experiments generally, and contributed to figure-making, tabulation and
    the writing of significant written portions of the text (notably, those sections dedicated to
    statistical matching, analysis of results, and, in the appendix, analysis of the learned
    matches and CaliperNN ablations).
    %
  \item 
    %
    \textbf{V. Sharmanska} contributed to the conceptual development of the proposed method and its
    predecessor, to the analysis of results, and to the general guidance of the project as a whole
    and the paper it begot (participating in regular discussion, giving constructive feedback on
    the paper and the rebuttal, etc.).
    %
  \item 
    %
    \textbf{N. Quadrianto} proposed the problem setting and the initial roadmap, co\"ordinated
    and supported team members, and supervised, and led discussion on, the project in all respects.
    %
    N. Quadrianto also contributed to the text itself in the drafting of the introduction and in
    providing comprehensive feedback on the state of the paper.
    %
\end{itemize}
%
}
