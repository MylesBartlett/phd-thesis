\epigraph{
    %
    \emph{
        %
        ``Begin at the beginning,'' the King said gravely, ``and go on till you come to the end:
        then stop.''
    %
} 
%
}
{Alice in Wonderland\\Lewis Carroll}
%
\noindent It is common knowledge that every good story should have a beginning, a middle, and an
end; for if a story had not those things it could scarcely be called a story at all, at most it
would be a nonsense one and nonsense should only be abided when founded on good sense.
%
This is where this thesis begins, a thesis that hopefully satisfies some of the reader's
sensibilities regarding what makes a good story; at the very least I hope it bears a scintilla of
sense, or perhaps even some estimate of erudition.
%
When I say \emph{story} I mean not, of course, to say that the contents is in any way
fictitious, counterfeit or embellished; as a story, this is a work of chapters and
bridging those chapters is evolution, both academically and personally.
%
For a story to \emph{become} -- to tell itself or let itself be told -- it must grow, by nature, by
contrivance, by necessity; stories reflect life and life is a process of growth, of betterment --
where each step moves us further along on a journey (not on a `chequerboard of nights and days', as
Khayyam so poetically but cynically scribed
%
\footnote{
    \emph{
        %
        “Tis all a Chequerboard of nights and days //
        %
        Where Destiny with men for Pieces plays: //
        %
        Hither and thither moves, and mates,and slays, //
        %
        And one by one back in the closet lays.”
        %
    }
}
%
), one without a destination, but a journey one should
never yield on regardless of the times one stumbles.
%
I would also call this thesis a \emph{story} for the simple reason that -- having childish
tendencies -- I am fond of stories and I should naturally like to like a work born from my hand and
essentia -- there are perhaps elements I should think less fondly of but might imagine (in the
style of some dualistic tale) as the shadow that makes the light -- the nobler qualities -- burn
all the brighter.

\section{Themes}
%
Every story has a theme, even if that theme is no theme at all, and whether it be conscious or
otherwise: an unbroken thread, the warp and weft, that weaves all into one.
%
This story is no exception; its theme is not one of valour, of defiance in the face of impossible
odds, or of taking the next step when the path ahead is fogged and the path behind beset with
demons -- indeed, the theme is nothing so inspiriting or ennobling -- that has the power to rouse
our best, eudaimonic selves -- but it bears its own importance -- not to the human condition but
the autonomous one -- nevertheless.





%


% \emph{
%     %
%     ``You might just as well say,'' added the March Hare, “that `I like what I get` is the same
%     thing as `I get what I like`!''
%     %
%     \\
%     %
%     ``You might just as well say,'' added the Dormouse, who seemed to be talking in his sleep,
%     ``that `I breathe when I sleep' is the same thing as `I sleep when I breathe'!'' 
%     %
% }

% \emph{
%     ``Would you tell me, please, which way I ought to go from here?''
%     \\
%     ``That depends a good deal on where you want to get to,'' said the Cat.
%     \\
%     ''I don’t much care where—'' said Alice.
%     \\
%     ``Then it doesn’t matter which way you go,'' said the Cat.
%     \\
%     ``-so long as I get somewhere,'' Alice added as an explanation.
%     \\
%     ``Oh, you’re sure to do that,'' said the Cat, ``if you only walk long enough.''
% }

% \emph{
%     %
%     The King’s argument was, that anything that had a head could be beheaded, and that you
% weren’t to talk nonsense.
%     %
% }

% \emph{
%     %
% ``Well! I’ve often seen a cat without a grin,'' thought Alice; ``but a grin without a cat! It's the
% most curious thing I ever saw in all my life!'' 
%     %
% }

% \emph{
%     %
%     I don’t see how he can ever finish, if he doesn't begin.
%     %
% }

% \emph{
% %
%     Alice didn't think that proved it at all; however, she went on ``And how do you know that
%     you’re mad?''
% %
%     ``To begin with,” said the Cat, ``a dog’s not mad. You grant that?''
% %
%     ``I suppose so,'' said Alice.
% %
%     ``Well, then,'' the Cat went on, ``you see, a dog growls when it's angry, and wags its tail
%     when it's pleased. Now I growl when I'm pleased, and wag my tail when I’m angry. Therefore I’m
%     mad.''
% %
% }


% \emph{
%     %
%     ``If there’s no meaning in it,'' said the King, ``that saves a world of trouble, you know, as
%     we needn't try to find any.''
%     %
% }
