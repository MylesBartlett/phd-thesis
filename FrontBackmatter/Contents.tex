%*******************************************************
% Table of Contents
%*******************************************************
\pagestyle{scrheadings}
%\phantomsection
\pdfbookmark[1]{\contentsname}{tableofcontents}
\setcounter{tocdepth}{1} % <-- 1 includes up to sections in the ToC
\setcounter{secnumdepth}{3} % <-- 3 numbers up to subsubsections
\manualmark
\markboth{\spacedlowsmallcaps{\contentsname}}{\spacedlowsmallcaps{\contentsname}}
\tableofcontents
\automark[section]{chapter}
\renewcommand{\chaptermark}[1]{\markboth{\spacedlowsmallcaps{#1}}{\spacedlowsmallcaps{#1}}}
\renewcommand{\sectionmark}[1]{\markright{\textsc{\thesection}\enspace\spacedlowsmallcaps{#1}}}

\makeatletter
\newcommand\invisiblechapter[1]{%
  \refstepcounter{chapter}%
  \addtocontents{toc}{\begingroup}%
        % \renewcommand{\cftchapaftersnumb}{\spacedlowsmallcaps}%
        % \renewcommand{\cftchapfont}{\normalfont}%
        % \renewcommand{\cftchappagefont}{\normalfont}%
        % }%
  % \addcontentsline{toc}{chapter}{\bfseries#1}%
  \addcontentsline{toc}{chapter}{\protect\numberline{\thechapter}\ct@caps#1}%
  \addtocontents{toc}{\endgroup}%
  \chaptermark{#1}}
\makeatother
%*******************************************************
% List of Figures and of the Tables
%*******************************************************
\clearpage
% \pagestyle{empty} % Uncomment this line if your lists should not have any headlines with section name and page number
\begingroup
  \let\clearpage\relax
  \let\cleardoublepage\relax
  %*******************************************************
  % List of Figures
  %*******************************************************
  % \pdfbookmark[1]{\listfigurename}{lof}
  % \listoffigures

  % \vspace{8ex}

  %*******************************************************
  % List of Tables
  %*******************************************************
  % \pdfbookmark[1]{\listtablename}{lot}
  % \listoftables

  % \vspace{8ex}

  %*******************************************************
  % List of Listings
  %*******************************************************
  % \pdfbookmark[1]{\lstlistlistingname}{lol}
  % \lstlistoflistings

  % \vspace{8ex}

  %*******************************************************
  % Glossary
  %*******************************************************
  % \pdfbookmark[1]{Glossary}{glossary}
  % \markboth{\spacedlowsmallcaps{Glossary}}{\spacedlowsmallcaps{Glossary}}
  % \chapter*{Glossary}
  % \printglossaries

  % \vspace{8ex}

  %*******************************************************
  % Acronyms
  %*******************************************************
  %\phantomsection

  \pdfbookmark[1]{Acronyms}{acronyms}
  \markboth{\spacedlowsmallcaps{Acronyms}}{\spacedlowsmallcaps{Acronyms}}
  \chapter*{Acronyms}
  \begin{acronym}[UMLX]
    \acro{AE}{autoencoder}
    \acro{AR}{acceptance rate}
    \acro{CNN}{Convolutional Neural Network}
    \acro{DL}{Deep Learning}
    \acro{DP}{demographic parity}
    \acro{EOdds}{equalised odds}
    \acro{EOpp}{equality of opportunity}
    \acro{ERM}{Empirical Risk Minimisation}
    \acro{GP}{Gaussian process}
    \acro{INN}{Invertible Neural Network}
    \acro{LR}{Logistic Regression}
    \acro{ML}{machine learning}
    \acro{MMD}{Maximum Mean Discrepancy}
    \acro{NN}{(artificial) neural network}
    \acro{SVM}{Support Vector Machine}
    \acro{TNR}{true negative rate}
    \acro{TPR}{true positive rate}
    \acro{VAE}{variational autoencoder}
    \acro{cVAE}{conditional VAE}
  \end{acronym}

  \vspace{8ex}
  % \clearpage

  %*******************************************************
  % Nomenclature
  %*******************************************************
  \pdfbookmark[1]{Nomenclature}{nomenclature}
  \markboth{\spacedlowsmallcaps{Nomenclature}}{\spacedlowsmallcaps{Nomenclature}}
  \begin{nomenclature}
    \entry{$P$}{Probability}
    \entry{$s$}{Sensitive attribute/spurious attribute/subgroup label}
    \entry{$S$}{Random variable for the sensitive attribute/spurious attribute/subgroup label}
    \entry{$\gS$}{Set of possible values for the sensitive attribute/spurious attribute/subgroup label}
    \entry{$\vx$}{Input features (without the \(s\) attribute)}
    % \entry{$\hat{\vx}$}{Reconstructed input}
    \entry{$y$}{Class label (ground truth)}
    \entry{$Y$}{Random variable for the class label}
    \entry{$\gY$}{Set of possible values for the class label}
    \entry{$\hat{y}$}{Predicted label}
    \entry{$\bar{y}$}{Fair target label}
    \entry{$\vz$}{Encoding of $\vx$}
  \end{nomenclature}

  \vspace{8ex}

  %*******************************************************
  % Glossary
  %*******************************************************
  \pdfbookmark[1]{Glossary}{glossary}
  \markboth{\spacedlowsmallcaps{Glossary}}{\spacedlowsmallcaps{Glossary}}
  \begin{glossaryenv}
    \entry{Adult/Census Income}{A popular fairness dataset based on census data from the U.S.}
    \entry{CelebA}{Face attributes dataset with more than 200K celebrity images}
    \entry{MNIST}{Dataset of handwritten digits}
    \entry{demographic group}{Set induced by the sensitive attribute}
    \entry{fairness definition}{An aspirational (often legally inspired) specification of a fair classifier}
    \entry{fairness metric}{A metric which quantifies how well a fairness definition is satisfied}
    \entry{sensitive attribute}{An attribute that, usually for legal or ethical reasons, should not be the basis for classification}
    \entry{spurious attribute}{An attribute that is correlated with the prediction target in the training set but not in the deployment setting}
    \entry{subgroup label}{A label indicating subgroups that should all be handled equally well by the classifier}
  \end{glossaryenv}



\endgroup
