%*******************************************************
% Table of Contents
%*******************************************************
\pagestyle{scrheadings}
%\phantomsection
\pdfbookmark[1]{\contentsname}{tableofcontents}
\setcounter{tocdepth}{1} % <-- 1 includes up to sections in the ToC
\setcounter{secnumdepth}{3} % <-- 3 numbers up to subsubsections
\manualmark
\markboth{\spacedlowsmallcaps{\contentsname}}{\spacedlowsmallcaps{\contentsname}}
\tableofcontents
\automark[section]{chapter}
\renewcommand{\chaptermark}[1]{\markboth{\spacedlowsmallcaps{#1}}{\spacedlowsmallcaps{#1}}}
\renewcommand{\sectionmark}[1]{\markright{\textsc{\thesection}\enspace\spacedlowsmallcaps{#1}}}

\makeatletter
\newcommand\invisiblechapter[1]{%
  \refstepcounter{chapter}%
  \addtocontents{toc}{\begingroup}%
        % \renewcommand{\cftchapaftersnumb}{\spacedlowsmallcaps}%
        % \renewcommand{\cftchapfont}{\normalfont}%
        % \renewcommand{\cftchappagefont}{\normalfont}%
        % }%
  % \addcontentsline{toc}{chapter}{\bfseries#1}%
  \addcontentsline{toc}{chapter}{\protect\numberline{\thechapter}\ct@caps#1}%
  \addtocontents{toc}{\endgroup}%
  \chaptermark{#1}}
\makeatother
%*******************************************************
% List of Figures and of the Tables
%*******************************************************
\clearpage
% \pagestyle{empty} % Uncomment this line if your lists should not have any headlines with section name and page number
\begingroup
  \let\clearpage\relax
  \let\cleardoublepage\relax
  %*******************************************************
  % List of Figures
  %*******************************************************
  % \pdfbookmark[1]{\listfigurename}{lof}
  % \listoffigures

  % \vspace{8ex}

  %*******************************************************
  % List of Tables
  %*******************************************************
  % \pdfbookmark[1]{\listtablename}{lot}
  % \listoftables

  % \vspace{8ex}

  %*******************************************************
  % List of Listings
  %*******************************************************
  % \pdfbookmark[1]{\lstlistlistingname}{lol}
  % \lstlistoflistings

  % \vspace{8ex}

  %*******************************************************
  % Glossary
  %*******************************************************
  % \pdfbookmark[1]{Glossary}{glossary}
  % \markboth{\spacedlowsmallcaps{Glossary}}{\spacedlowsmallcaps{Glossary}}
  % \chapter*{Glossary}
  % \printglossaries

  % \vspace{8ex}

  %*******************************************************
  % Acronyms
  %*******************************************************
  %\phantomsection

  \pdfbookmark[1]{Acronyms}{acronyms}
  \markboth{\spacedlowsmallcaps{Acronyms}}{\spacedlowsmallcaps{Acronyms}}
  \chapter*{Acronyms}
  %
  \begin{acronym}[UMLX]
    %
    \acro{AE}{Auto-encoder}
    \acro{AF}{Algorithmic Fairness}
    \acro{AdvL}{Adversarial Learning}
    \acro{CBN}{Causal Bayesian Network}
    \acro{CNN}{Convolutional Neural Network}
    \acro{DAG}{Directed Acyclic Graph}
    \acro{DA}{Domain Adaptation}
    \acro{DG}{Domain Generalisation}
    \acro{DL}{Deep Learning}
    \acro{DNN}{Deep Neural Network}
    \acro{DP}{Demographic Parity}
    \acro{DRO}{Distributional Robust Optimisation}
    \acro{DR}{Distributional Robustness}
    \acro{ELBO}{Evidence Lower Bound}
    \acro{ERM}{Empirical Risk Minimisation}
    \acro{ER}{Equalised Rate}
    \acro{EqOd}{Equalised Odds}
    \acro{EqOp}{Equality of Opportunity}
    \acro{FPR}{False Positive Rate}
    \acro{FRL}{Fair-Representation Learning}
    \acro{GAN}{Generative Adversarial Network}
    \acro{HGRMC}{Hirschfeld-Gebe\-lein-R\'enyi Maximal Correlation}
    \acro{ID}{In-Distribution}
    \acro{IF}{Individual Fairness}
    \acro{INN}{Invertible Neural Network}
    \acro{IW}{Importance Weight}
    \acro{KL}{Kullback-Leibler}
    \acro{MI}{Mutual Information}
    \acro{MLE}{Maximum Likelihood Estimation}
    \acro{MLP}{Multi-Layer Perceptron}
    \acro{ML}{Machine Learning}
    \acro{MMD}{Maximum Mean Discrepancy}
    \acro{MOO}{Multi-objective Optimisation}
    \acro{MS}{Missing Sources}
    \acro{NF}{Normalising Flow}
    \acro{OOD}{Out-of-Distribution}
    \acro{PF}{Pareto Frontier}
    \acro{PO}{Pareto Optimal}
    \acro{RLAIF}{Reinforcement Learning from AI Feedback}
    \acro{RLHF}{Reinforcement Learning from Human Feedback}
    \acro{RobAcc}{Robust Accuracy}
    \acro{SCL}{Shortcut Learning}
    \acro{SC}{Spurious Correlation}
    \acro{SGD}{Stochastic Gradient Descent}
    \acro{SL}{Supervised Learning}
    \acro{SelfSL}{Self-Supervised Learning}
    \acro{SemiSL}{Semi-Supervised Learning}
    \acro{TNR}{True Negative Rate}
    \acro{TPR}{True Positive Rate}
    \acro{UDA}{Unsupervised Domain Adaptation}
    \acro{UL}{Unsupervised Learning}
    \acro{VAE}{Variational Auto-encoder}
    \acro{cFlow}{conditional Flow}
    \acro{cVAE}{conditional VAE}
    %
  \end{acronym}

  % \vspace{8ex}
  % \clearpage
  \newpage

  %*******************************************************
  % Nomenclature
  %*******************************************************
  % \pdfbookmark[1]{Nomenclature}{nomenclature}
  % \markboth{\spacedlowsmallcaps{Nomenclature}}{\spacedlowsmallcaps{Nomenclature}}
  % \begin{nomenclature}
  %   \entry{$P$}{Probability}
  %   \entry{$s$}{Sensitive attribute/spurious attribute/subgroup label}
  %   \entry{$S$}{Random variable for the sensitive attribute/spurious attribute/subgroup label}
  %   \entry{$\gS$}{Set of possible values for the sensitive attribute/spurious attribute/subgroup label}
  %   \entry{$\vx$}{Input features (without the \(s\) attribute)}
  %   % \entry{$\hat{\vx}$}{Reconstructed input}
  %   \entry{$y$}{Class label (ground truth)}
  %   \entry{$Y$}{Random variable for the class label}
  %   \entry{$\gY$}{Set of possible values for the class label}
  %   \entry{$\hat{y}$}{Predicted label}
  %   \entry{$\bar{y}$}{Fair target label}
  %   \entry{$\vz$}{Encoding of $\vx$}
  % \end{nomenclature}

  % \vspace{8ex}
  % \clearpage
  \newpage

  %*******************************************************
  % Glossary
  %*******************************************************
  \pdfbookmark[1]{Glossary}{glossary}
  \markboth{\spacedlowsmallcaps{Glossary}}{\spacedlowsmallcaps{Glossary}}
  \begin{glossaryenv}
    %
    \entry{CelebA}{
        %
      A popular benchmark computer-vision dataset comprising more than 200K celebrity head-shots
      annotated with various physical and affective attributes, such as `Smiling', `Gender', and
      `Age'.
      %
      The dataset has cemented itself as one of the principal benchmark datasets in the \ac{AF} and
      \ac{DRO} literature due to the \acp{SC} consequent of its (conditional) label-imbalance (
      `Blond' individuals being predominantly female, for instance).
      %
    } 
    %
    \entry{Domain}{ 
      %
      The 'source' of a particular subset of the data, each domain assumed to embody a different
      sub-distribution of the collective data induced by, for instance, variations in recording
      equipment, category (in the context of sentiment analysis), environs, geography (in the
      context of remote sensing), and weather conditions. 
      %
      Thus, like with sensitive attribute, the domain imposes some secondary structure on the data
      and in the context of \ac{DA} and \ac{DG} I will talk of desiring invariance to this
      structure, a desideratum that aligns with that of \ac{AF} (at least in certain senses).
      %
      I would note that I also use, with some frequency, \textbf{domain} in the functional-analysis
      sense, referring to the space of expected input values (correlatively with codomain),
      however, it should hopefully always be clear from the context which sense I am invoking.
    %
    } 
    %
    \entry{Environment}{
      %
      See \textbf{domain}.
      %
    }.
    %
    \entry{MNIST}{
      \noindent
      %
      A dataset of handwritten digits (itself derived from a larger dataset known as the NIST
      Special Database) and a foundational computer-vision benchmark that continues to see use even
      to this day under a variety guises, despite its simplicity (this simplicity, making it
      well-suited for demonstrating feasibility).
      %
      In this thesis I will talk of a colourised version of the dataset, in both
      Chapters~\ref{ch:nifr} and \ref{ch:supmatch} -- variants of which have seen widespread
      adoption in the \ac{DG} literature, most notably in \citep{arjovsky2019invariant} -- where
      the colourisation is performed correlatively with the digit labels so as to induce a \ac{SC}.
      %
    } \entry{Protected attribute}{
      %
      See \textbf{sensitive attribute}.
      %
    } \entry{Sensitive attribute}{
      %
      An attribute of the data that, for legal or ethical reasons, should not be factored into the
      predictions of a model, and \wrt{} fairness metrics are computed to judge the degree of this
      violation.
    %
    } \entry{Subgroup}{
      %
      The general term  (void of any subfield-specific connotations) for sensitive attribute and
      domain (and, transitively, their synonyms), and thus referring to a given secondary group,
      partition, or sub-distribution of the data.
    %
    }

    \entry{UCI Adult (Income)}{
      %
      A popular tabular \ac{AF} dataset derived from U.S. census data; the canonical task is to
      predict whether an individual earns $\$50$K or more (positive class) or not (negative class)
      and the sensitive attribute, \wrt{} which fairness is computed, is typically taken to be
      `Sex' `Race', or `Age`.
      %
    } \entry{WILDS}{
      %
      A suite of datasets (spanning various tasks, domains, and modalities) for evaluating the
      distributional-robustness/\ac{DG} capabilities of models under in-the-wild (real-world)
      distributions shifts.
      %
    }
  %
  \end{glossaryenv}


\endgroup
