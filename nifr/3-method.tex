\section{Interpretable invariances by null-sampling}\label{ssec:general}
\subsection{Problem Statement}
%
\noindent
%
We assume we are given inputs $x \in \gX$ and corresponding labels $y \in \gY$.
%
Furthermore, there is some spurious variable $s \in \gS$ associated with each input $x$ which
we do \emph{not} want to predict. 
%
Let $X$, $S$ and $Y$ be the random variables that take on the observed values $x$, $s$ and $y$,
respectively. 
%
The fact that both $Y$ and $S$ are predictive of $X$ implies that $\gI(X;Y), \gI(X;S) > 0$, where
$\gI(\cdot ;\cdot)$ denotes \ac{MI} between two random variables.
%
Note, however, that the conditional entropy is non-zero: \( H(S|X) > 0 \), i.e.\ $S$ is \emph{not}
completely determined by $X$.

%The difficulty emerges in the construction of the fully-supervised training dataset in which
%correspondence between $S$ and $Y$ is exaggerated compared to the test set.
The difficulty of this setup resides in the fact there is a close correspondence between $S$ and
$Y$ in the training set such that for a classifier trained via maximum-likelihood estimation, the
mappings \( \gX \to \gS \) and \(\gX \to \gY \) are functionally equivalent, which implies, through
transitivity, that \(\gX \to \gS \to \gY \) also is; in many cases, such as those we consider in
this paper, the first part of the chain, \( \gX \to \gY \), is substantially easier to learn than
the direct, and, importantly, \emph{causal}, path.
%
This is problematic when we assume that the same correlation does \emph{not} hold in the test set,
meaning the model cannot rely on shortcuts provided by $S$ if it is to generalise from the training
set.

We call this scenario where we only have access to the labels of a biasedly-sampled subpopulation
an \emph{aggravated fairness problem}; scenarios of this nature are not uncommon in the real-world. 
%
For instance, in long-feedback systems such as mortgage-approval where the demographics of the
subpopulation with observed outcomes is \emph{not} representative of the subpopulation on which the
model has been deployed. 
%
In this case, $s$ has the potential to act as a false (or \emph{spurious}) indicator of the class
label and training a model with such a dataset would limit generalisability. 
%
Let \( (X^{tr}, S^{tr}, Y^{tr}) \) then be the random variables sampled from the training set, and
\( (X^{te}, S^{te}, Y^{te}) \) likewise be the random variables sampled from the test set.
%
The training and test sets thus induce the following inequality for their \ac{MI}:
\( \gI(S^{tr}; Y^{tr}) \gg \gI(S^{te}; Y^{te}) \approx 0 \).

Our goal is to learn a representation $Z_u$ (with realisations \(z_u\)), that is independent of $S$
and transferable between downstream tasks. 
%
Complementary to $z_u$, we refer to some abstract component of the model that absorbs the unwanted
information related to $S$ as $\gB$, the realisation of which we define \wrt{} each of the two
models to be described.
%To satisfy this objective, we introduce an additional regularisation term that can be viewed from
%an information-theoretic perspective as minimising the mutual information between the random
%variables:
The requirement for $Z_u$ can be expressed in terms of \ac{MI} as
%
\begin{align}
  \gI(Z_u; S) \neq 0~.
  \label{eq:migoal}
\end{align}
%
However, for the representation to be useful, we need to capture as much semantically-relevant
information from the data as possible. 
%
Incorporating this requirement naturally gives rise to the following objective function
%
\begin{align}
  \min_{\theta}
  \E_{(X,S) \sim P^{tr}_{(X, S)}} [
  \lambda \gI(f_\theta(X);S) -\log p_\theta(X) 
  ],
  \label{eq:objectivetheory}
\end{align}
%
where $\theta$ refers to the trainable parameters of our model, \( f_\theta \), and \(
p_\theta(\cdot) \) is the likelihood it assigns to the training data, and \( P^{tr}_{XS} \) denotes
the joint distribution over \( X^{tr} \) and \( S^{tr} \).
%
Note that we have slightly abused notation here in allowing \(f\) (a Borel Measurable function) to
accept random variables \(X\) and thereby output random variables, \(Z_u\); the mapping \(f(X)\)
should be understood to mean \( f \circ X(\omega) \) for some event \( \omega \in \Omega \), on
which basis \( f(x) \) can be reinterpreted as \( f(X=x) \).
%
In practice, we optimise this loss in an adversarial fashion by playing a minimax game, in which
our encoder acts as the generative component from a \ac{GAN} \citep{goodfellow2014generative}
perspective.
%
The adversary is an auxiliary classifier \(g: \to \bigtriangleup^{|\gS|} \) trained to predict the
spurious variable \(s\) from \(z_u\), with \(\bigtriangleup^{|\gS|}\) being the probability simplex
over \(\gS\).
%
We denote the trainable parameters of the adversary as $\phi$; for the parameters of the encoder we
use $\theta$, as before. 
%
The theoretical objective from Eq.~\ref{eq:objectivetheory} can then be crystallised as
%
\begin{align}
  %
  \min_{ \theta \in \Theta} \max_{\phi \in \Phi}
  \E_{(x, s) \sim P^{tr}_{(x,s)}}[
  \log p_\theta(x)
  -\lambda H( g_\phi ( f_\theta(x) ), e_{s})
  ],
  \label{eq:objectivepractical}
  %
\end{align} 
%
where we have substituted \( P^{ tr }_{ (X, S) } \) with the empirical training distribution \( P^{
tr }_{ (x, s) } \), and \( H(\cdot, \cdot) \) denotes the cross-entropy between the predicted
probabilities and the degenerate target distribution given by the one-hot-encoded labels, $e_{s}
\in \{0, 1\}^{|\gS|}$.
%
In practice, this adversarial term is realised using a Gradient-Reversal Layer (GRL;
\citealp{ganin2016domain}) between \(z_u\) and \(g\), as is common for adversarial approaches for
\acl{DA} and \acl{FRL}~\citep{edwards2016censoring}.
%
\subsection{The Disentanglement Dilemma}
%
The objective in~\eqref{eq:objectivepractical} balances the two desiderata: predicting $y$ and
being invariant to $s$.
%
However, in the training set, $y$ and $s$ are so strongly correlated that removing information
about $s$ implies removing information about $y$, causing existing methods to fail under this
setting.
% However, this objective is complicated by the desideratum that $z_u$ remain predictive of $y$,
% which precludes us from directly training on the target-labelled dataset $(X^{tr}, S^{tr},
% Y^{tr})$,
%where $y$ and $s$ are so strongly correlated that removing information about $s$ inevitably
%removes information about $y$. We therefore need
In order to even define a well-posed learning objective, we require another source of information that
allows us to disentangle $s$ and $y$.
%
For this, we assume the existence of another set of samples that follow a similar distribution to
the test set, but while the sensitive attribute is available, the class labels are not. 
%
In reality, this is not an unreasonable assumption, as, while properly annotated data is scarce,
unlabelled data can be obtained in abundance (with demographic information from census data,
electoral rolls, etc.).
%
Indeed, treating the data as unlabelled only \wrt{} \(y\), with the $s$ labels intact, is not
without precedence in the fairness literature (\citealp{wick2019unlocking, creager2019flexibly}, inter alia).
%
We are restricted only in the sense that the spurious correlations we want to sever are indicated
in the features.
%
We call this the \emph{representative set}, with random variables $X^{rep}$ and \( S^{rep} \) and
satisfying the condition that \( \gI(S^{rep}; Y^{rep}) \approx 0 \) (or rather, it would if the
class labels \( Y^{rep} \) were available).

We now summarise the training procedure; an outline of the invertible network model (\acs{cFlow})
can be seen in Fig.~\ref{fig:inn_diagram}.
%
\import{nifr}{model-diag.tex}
%
Textually, first, the encoder network $f$ is trained on \( (X^{rep}, S^{rep}) \), during the first
phase.
%
The trained network is then used to encode the training set, taking in input $x$ and producing the
representation, $z_u$, decorrelated from the spurious variable.
%
The encoded dataset can then be used to train any off-the-shelf classifier safely, with information
about the spurious variable having been absorbed by some auxiliary component $\gB$.
%
In the case of the \acf{cVAE} model, $\gB$ takes the form of the decoder subnetwork, which
reconstructs the data conditional on a one-hot encoding of $s$, while for the invertible network
$\gB$ is realised as a partition of the feature map $z$ (such that $z \triangleq [z_u, z_b]$,
where \( [\cdot] \) denotes concatenation), given the bijective constraint.
%
Thus, the classifier cannot take the shortcut of learning $s$ and instead must learn how to predict
$y$ directly.
%
Obtaining the $s$-invariant representations, $x_u$, in the data domain is simply a matter of
replacing the $\gB$ component of the decoder's input for the \ac{cVAE}, and $z_b$ for
\ac{cFlow}, with a zero vector of equivalent size.
%
We refer to this procedure used to generate $x_u$ as \emph{null-sampling} (here, with respect
to $z_b)$.

% This That said, we do wish to draw a distinction between null-sampling and the annihilation
% operation featured in .
Null-sampling resembles the \emph{annihilation} operation described in \citet{xiao2017dna}, however
we note that the two serve very different roles.  
%
Whereas the annihilation operation serves as a regulariser to prevent trivial solutions (similar to
\citealp{jaiswal2018unsupervised}), null-sampling is used to generate the invariant representations
post-training.

\subsection{Conditional Decoding}%
%
\label{conddec}
\noindent We first describe a \acs{VAE}-based model similar to that proposed
in~\citet{madras2018learning}, before highlighting some of its shortcomings that motivate the
choice of an invertible representation learner.

The model takes the form of a class conditional $\beta$-\acs{VAE} \citep{higgins2017beta}, in which the
decoder is conditioned on the spurious attribute. 
%
We use $\theta_{enc}, \theta_{dec} \in \theta$ to denote the parameters of the encoder and decoder
sub-networks, respectively. 
%
Concretely, the encoder component performs the mapping $x \to{z_u}$, while $\gB$ is instantiated as
the decoder, $\gB \triangleq p_{\theta_{dec}}(x|z_u, s)$, which takes in a concatenation of the
learned non-spurious latent vector $z_u$ and a one-hot encoding of the spurious label $s$ to
produce a reconstruction of the input $\hat{x}$. 
%
Conditioning on a one-hot encoding of $s$, rather than a single value, as done in
\citet{madras2018learning}, is the key to visualising invariant representations in the data domain.
%
If $\gI(z_u; s)$ is properly minimised, the decoder can only derive its information about $s$ from
the label, thereby freeing up $z_u$ from encoding the unwanted information while still allowing for
reconstruction of the input.
%
Thus, by feeding a zero-vector, \(e_s\), to the decoder we achieve $\hat{x} \perp s$. 
%
The full learning objective for the \ac{cVAE} is given as
%
\begin{align}
%
\begin{split}
    %
    \gL_{\mathrm{cVAE}} =& 
    \E_{q_{\theta_{enc}}(z_u|x)}[
    \log
    p_{\theta_{dec}}(x|z_u, e_s) - \log p_{\theta_{dec}}(s|z_u)
    ] \\ &- \beta \KL(q_{\theta_{enc}}(z_u |x) \| p(z_u)),
    %
\end{split}
%
\label{eq:cvae-obj}
\end{align}
%
where $\beta$ is a hyperparameter that determines the trade-off between reconstruction accuracy and
independence constraints, and $p(z_u)$ is the prior imposed on the variational posterior. 
%
For all our experiments, $p(z_u)$ is the standard isotropic Gaussian prior;
Fig.~\ref{fig:cvae_diagram} summarises the procedure diagrammatically.
%
% CORRECTION: if you've put it in you should explain it

While we show this setup can indeed work for simple problems, as~\citet{madras2018learning} before
us have, we show that it lacks scalability due to conflict between the components of the loss.
%
\corr{
  %
Since information about $s$ is only available to the decoder as a binary encoding, should \(s\) be
fundamentally continuous in nature -- as is so in the case of gender, for individuals exhibit
varying degrees of femininity/masculinity -- and said encoding a coarsened representation of that
nature, off-loading information entirely to the decoder by way of conditioning is no longer a valid
strategy. 
%
}
% CORRECTED: see discussion in viva
%
As a result, $z_u$ is forced to carry information about $s$ in order to minimise the
reconstruction error. 

The obvious solution to this is to allow the encoder to store information about $s$ in a partition
of the latent space as in  \citet{creager2019flexibly}. 
%
However, we question whether an \ac{AE} is the best choice for this setup, and that \acp{INN} might
not be in inherently more well-positioned for such.
%
Indeed, \acp{INN} afford several guarantees, most germane among these being complete
information-preservation and freedom from a reconstruction loss, the importance of which we
expatiate on in the following section.

\subsection{Conditional Flow}\label{cflow}
%
\paragraph{Invertible Neural Networks.}
%
\Acp{INN} embody a subclass of neural networks characterised by a bijective mapping between their
inputs and output \citep{Dinh2014}. 
%
The transformations are designed such that their inverses and Jacobians are exactly and efficiently
computable.
%
These flow-based models permit \emph{exact} likelihood estimation \citep{normflows2015} through the
warping of a base density with a series of invertible transformations and computing the resulting,
highly multi-modal, but still normalised, density, using the change-of-variable theorem:
% Flow-GAN \cite{grover2018flowgan} combines the \emph{exact} log-likelihood estimation of the
% invertible network with the adversarial training of a GAN.
%
\begin{align}
\begin{split}
  \log p(x) &= \log p(z) + 
   \sum \log \left| \det\left( \frac{\diff h_i}{ h_{i-1}}\right) \right|, %\\
  \quad p(z) = \gN(z; 0, \sI),
  \label{eq:changeofvariables}
\end{split}
\end{align}
%
where $h_i$ denotes the output of the \(i\)th layer of the network and $p(z)$ is the base density,
which is again an isotropic Gaussian. 
%
Training the \ac{INN} then reduces to maximising $\log p(x)$ over the training set, i.e.\ maximising the
probability the network assigns to samples in the training set.
%
\paragraph{The Benefits of Bijectivity.}
%
Using an \ac{INN} to generate our encoding, $z_u$, carries a number of advantages over other
approaches. 
%
Ordinarily, the main benefit of flow-based models is that they permit exact density estimation.
%
However, since we are not interested in sampling from the model's distribution, in our case the
likelihood term serves as a regulariser, in the same vein as \citet{JacSmeOya18}.
%
Critically, this forces the mean of each latent dimension to zero, thereby enabling null-sampling.
%
The invertible property of the network guarantees the preservation of all information relevant to
$y$ which is independent of $s$, regardless of how it is allocated in the output space. 
%
\corr{
  %
Secondly, we hypothesise that, due to the this information-preserving property, the encodings
should be more robust to \ac{OOD} data; we provide preliminary empirical evidence for this in
Appendix~\ref{sec:transfer-learning}.
  %
}
%
% CORRECTED: Say where the evidence for this is in the thesis eg fig 3.12. %
%
Whereas an \acf{AE} could map a previously seen input and a previously unseen input to the same
representation, an invertible network sidesteps this due to the network's bijective property,
ensuring all relevant information is stored somewhere. 
%
This opens up the possibility of transfer learning between datasets with a similar manifestation of
$s$, as we demonstrate \S\ref{sec:transfer-learning}.

Under our framework, the invertible network $f$ maps the inputs $x$ to a representation
$z \triangleq f(x)$.
%
We interpret the representation $z$ as being the concatenation of two subembeddings, namely \( z
\triangleq [z_u, z_b] \). 
%
The dimensionality of $z_b$ (and $z_u$, by complement) is a free parameter (see
\S\ref{sec:nifr-optimisation-details} for tuning strategies). 
%
As $f$ is invertible, $x$ can be recovered as such:
%
\begin{align}
  x = f^{-1}([z_u, z_b])
  \label{eq:zreconstruct}
\end{align}
%
where $z_b$ is required for equality of the output dimension and input dimension to satisfy
the bijectivity of the network -- we cannot output $z_u$ alone, but have to output $z_b$
as well. 
%
In order to generate the pre-image of $z_u$, we perform null-sampling \wrt{} $z_b$ by zeroing-out
the elements of $z_b$ (such that $x_u \triangleq f^{-1}([z_{u}, \vzero])$), i.e.\ setting them to
the mean of the prior density, $\gN(z;0, I)$.

How can ensure that $z_u$ contains the information about $y$ necessary for downstream
classification?
%
The importance of the invertible architecture bears out from this consideration, for so long as
$z_b$ does not contain the information about $y$, $z_u$ necessarily must.
%
We can then raise or lower the information capacity of $z_b$ by adjusting its dimensionality, \(
\text{dim}(z_b) \); practically, it should be set to the smallest size sufficient to capture all
information about $s$, so as not to sacrifice class-relevant information. 
%
\S\ref{sec:additional-results} explores the influence of \( \mathrm{dim}(z_b) \) empirically.
%
% Eq~\eqref{eq:zreconstruct} defines how to obtain $x$. In order to generate the pre-image of
% $z_u$, we perform null-sampling with respect to $z_b$ by zeroing-out its elements --
% i.e. setting them to the mean of the prior density imposed on $z$, $\mathcal{N}(z;0, I)$ --
% by the operation, $x_{u} = f^{-1}([z_{u}, \stackrel{\rightarrow}{0}])$.

% \paragraph{Preprocessing}. Heuristically, we found that preprocessing the data with an
% autoencoder stabilises and accelerates training of the cFlow model. The autoencoder was
% pretrained on the pretraining set solely to minimise reconstruction loss and its weights frozen
% at the time of the INN's training. While this means the INN is not truly lossless with respect to
% the uncompressed data, its bijectivity is leveraged to ensure semantically-relevant information
% is not discarded during the pre-training phase, which is still applicable since the autoencoder
% is not trained jointly with the INN to maximise the adversarial loss. Since the autoencoder is
% optimised for compression, information about both the spurious and non-spurious attributes is
% captured impartially in its encoding.
