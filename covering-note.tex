\section*{List of corrections}
%
\begin{itemize}
    %
    \item All typographic and notational errors highlighted by the examiners have been amended.
    %
    \item (pp.6,7) Footnotes with references to the publications underpinning Chapters 3 and 5 have
        been added to the chapter summaries at the end of the Introduction.
    %    
    \item (p.19) CelebA example provided in viva has been added to clarify the failure mode of DANN
        and related methods arising due to correlation between the target and subgroup (domain)
        attribute.
    %
    \item (p.49) Added conjecture on the sub-ERM performance of the Ln2L baseline, namely it being
        a result of the method's adversarial mutual-information-minimisation driving the encoder
        towards a digit-invariant solution on account of the strong spurious correlation.
    %
    \item (p.51) Fig. 3.1 has been relocated to a point adjacent to the in-text first reference to
        it. The caption has been greatly expanded, now addressing the involvement of null-sampling
        during training (i.e. none), this being a major point of confusion raised by the external
        examiner during the viva.
    %
    \item (p.53) Added explanation about the sub-baseline accuracy of our models being a natural
        consequence of the accuracy-fairness trade-off.
    %
    \item (p.70) Personal contributions to Chapter 3 have been elaborated on.
    %
    \item (p.83) The use of the mean of the projected representations as queries in the bag-wise
        aggregation function has been clarified in the main text. A note on the instance-wise and
        bag-wise objectives having theoretically- but not empirically-equivalent solutions (as
        adverted to by the external examiner) has been added.
    %
    \item (p.53) Statement on the shortcomings of relying on a binary mechanism when the underlying
        subgroup attribute is continuous has been clarified/reworked -- the reference to
        non-linearity has been removed due to its being misleading.
    %
    \item (p.54) A reference to the appendix providing (preliminary) empirical evidence of the
        hypothesis of increased OOD-robustness has been included here (together with some
        rephrasing).
    %
    \item (pp.90,91) Conclusion to Chapter 3 has been augmented with details of ongoing work, work
        that includes efforts to shore up the paucity/syntheticity of the tested-on datasets as
        well as conduct sensitivity analyses \wrt{} bag-balancing.
    %
    \item (p.119) The caption of Fig. 5.1 has been greatly expanded to now contain summaries of the
        two calipers and explanation of the red crosses (denoting exclusion from the candidate
        pool) added over the course of the pipeline. Readers in search of more details regarding
        the calipers and their motivation are referred to the RealPatch paper introducing them.
    %
    \item (p.128) Added paragraph to limitations section of Discussion noting the
        syntheticity/paucity of datasets experimented with in Chapters 3 and 4.
    %
\end{itemize}
